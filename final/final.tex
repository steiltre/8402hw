%        File: final.tex
%     Created: Thu May 05 09:00 PM 2016 C
% Last Change: Thu May 05 09:00 PM 2016 C
%

\documentclass[a4paper]{article}

\title{Math 8402 Take-Home Final }
\date{5/13/16}
\author{Trevor Steil}

\usepackage{amsmath}
\usepackage{amsthm}
\usepackage{amssymb}
\usepackage{esint}

\newtheorem{theorem}{Theorem}[section]
\newtheorem{corollary}{Corollary}[section]
\newtheorem{proposition}{Proposition}[section]
\newtheorem{lemma}{Lemma}[section]
\newtheorem*{claim}{Claim}
\newtheorem*{problem}{Problem}
%\newtheorem*{lemma}{Lemma}
\newtheorem{definition}{Definition}[section]

\newcommand{\R}{\mathbb{R}}
\newcommand{\N}{\mathbb{N}}
\newcommand{\C}{\mathbb{C}}
\newcommand{\Z}{\mathbb{Z}}
\newcommand{\supp}[1]{\mathop{\mathrm{supp}}\left(#1\right)}
\newcommand{\lip}[1]{\mathop{\mathrm{Lip}}\left(#1\right)}
\newcommand{\curl}{\mathrm{curl}}
\newcommand{\la}{\left \langle}
\newcommand{\ra}{\right \rangle}
\renewcommand{\vec}[1]{\mathbf{#1}}

\newenvironment{solution}{\emph{Solution.}}

\begin{document}
\maketitle
\begin{enumerate}
  \item
    \begin{problem}
      Consider the following equation:
      \begin{equation}
        \frac{\partial u}{\partial t} = \frac{\partial}{\partial x} \left( \frac{1}{\varepsilon} b \left( \frac{x}{\varepsilon} \right) u + D
        \frac{\partial u}{\partial x} \right)
        \label{eqn:prob1_1}
      \end{equation}
      where $D>0$ and $b(y)$ is a periodic function in $y$ with period 1 whose average is equal to 0:
      \begin{equation}
        \int_{0}^{1} b(y) dy = 0
        \label{eqn:prob1_2}
      \end{equation}

      We will be interested in positive soltuions ($u>0$). Assume that $u(x,t)$ can be written as:
      \begin{equation}
        u(x,t) = u(x,y,t) = u_0(x,y,t) + \varepsilon u_1(x,y,t) + \varepsilon^2 u_2(x,y,t) + \dots
        \label{eqn:prob1_3}
      \end{equation}
      where $y = \frac{x}{\varepsilon} $.

      \begin{enumerate}
        \item Show that $u_0(x,y,t)$ can be written as:
          \begin{equation}
            u_0(x,y,t) = \widehat{u} (x,t) v_0(y)
            \label{eqn:prob1_4}
          \end{equation}
          where $v_0(y)$ is a function that does not depend on $x$ or $t$.

        \item Show that $\widehat{u}(x,t)$ satisfies the equation:
          \begin{equation}
            \frac{\partial \widehat{u}}{\partial t} = \widehat{D} \frac{\partial^2 \widehat{u}}{\partial x^2} .
            \label{eqn:prob1_5}
          \end{equation}
          Obtain an explicit expression for $\widehat{D}$ and show that $\widehat{D}$ is never greater than $D$.

      \end{enumerate}

    \end{problem}

    \begin{solution}

      \begin{enumerate}
        \item
          Because $y = \frac{x}{\varepsilon}$, we write our equation as
          \[ \frac{\partial u}{\partial t} = \left( \frac{\partial}{\partial x} + \frac{1}{\varepsilon} \right) \left( \frac{1}{\varepsilon} b(y) u +
          D \left( \frac{\partial}{\partial x} + \frac{1}{\varepsilon} \frac{\partial}{\partial y} \right) u \right) .\]

          By expanding $u$ asymptotically, we get the $O \left( \frac{1}{\varepsilon^2} \right)$ equation
          \begin{align*}
            0 &= \frac{\partial}{\partial y} ( b(y) u_0 ) + \frac{\partial}{\partial y} \left( D \frac{\partial}{\partial y} u_0 \right) \\
            &= \frac{\partial}{\partial y} \left( b(y) u_0 + D \frac{\partial}{\partial y} u_0 \right)
          \end{align*}
          Therefore,
          \[ b(y) u_0 + D \frac{\partial}{\partial y} u_0 = c(x,t) \]
          for some function $c$.

          Then finding an integrating factor gives
          \begin{equation}
            \frac{\partial}{\partial y} \left( u_0 \exp \left( \frac{1}{D} \int_{0}^{y} b(z) dz \right) \right) = c(x,t)
            \exp \left( \frac{1}{D} \int_{0}^{y} b(z) dz \right) .
            \label{eqn:sol1_1}
          \end{equation}

          Integrating from 0 to 1, we get
          \begin{align*}
            0 &= u_0(1) - u_0(0) \quad \parbox{5cm}{by periodicity} \\
            &= u_0(y) \exp \left( \frac{1}{D} \int_{0}^{1} b(z) dz \right) \bigg|_0^1 \quad \parbox{4cm}{because $b$ has average zero} \\
            &= \int_{0}^{1} \frac{\partial}{\partial y} \left( u_0 \exp \left( \frac{1}{D} \int_{0}^{y} b(z) dz \right) \right) dy \quad
            \parbox{5cm}{by \eqref{eqn:sol1_1}} \\
            &= c(x,t) \int_{0}^{1} \exp \left( \frac{1}{D} \int_{0}^{y} b(z) dz \right) dy
          \end{align*}

          Because $e^x > 0$ for all $x$, we must have $c(x,t) = 0$. Plugging this into \eqref{eqn:sol1_1}, we have
          \[ \frac{\partial}{\partial y} \left( u_0 \exp \left( \frac{1}{D} \int_{0}^{y} b(z) dz \right) \right) = 0 .\]

          Therefore,
          \[ u_0 \exp \left( \frac{1}{D} \int_{0}^{y} b(z) dz \right) = \widehat{u} (x,t) \]
          for some $\widehat{u}$. Thus,
          \[ u_0 = \widehat{u}(x,t) v_0(y) \]
          where
          \[ v_0(y) = \exp \left( - \frac{1}{D} \int_{0}^{y} b(z) dz \right) .\]

        \item
          We have the $O \left( \frac{1}{\varepsilon} \right)$ equation
          \[ 0 = \frac{\partial}{\partial y} \left( b u_1 + D \frac{\partial u_1}{\partial y} \right) + \frac{\partial}{\partial x} \left( b u_0 + 2D
          \frac{\partial u_0}{\partial y} \right) .\]

          Substituting $u_0 = \widehat{u} (x,t) v_0(y)$, we have
          \begin{equation}
            \frac{\partial}{\partial y} \left( b u_1 + D \frac{\partial u_1}{\partial y} \right) = - \frac{\partial}{\partial x} \left( b(y)
            \widehat{u}(x,t) v_0(y) + 2 D \widehat{u}(x,t) \frac{\partial v_0}{\partial y} \right) .
            \label{eqn:sol1_2}
          \end{equation}

          Using the definition of $v_0$, we know
          \begin{align*}
            \frac{\partial v_0}{\partial y} &= \frac{\partial}{\partial y} \left[ \exp \left( -\frac{1}{D} \int_{0}^{y} b(z) dz \right) \right] \\
            &= -\frac{1}{D} b(y) \exp \left( -\frac{1}{D} \int_{0}^{y} b(z) dz \right) \\
            &= -\frac{1}{D} b(y) v_0
          \end{align*}

          Substituting this into \eqref{eqn:sol1_2}, we get
          \[ \frac{\partial}{\partial y} \left( b u_1 + D \frac{\partial u_1}{\partial y} \right) = -D \frac{\partial v_0}{\partial y}
          \frac{\partial \widehat{u}}{\partial x} \]

          Taking antiderivatives on both sides give
          \begin{equation}
            b u_1 + D \frac{\partial u_1}{\partial y} = - D v_0 \frac{\partial \widehat{u}}{\partial x} + \tilde{c}(x,t)
            \label{eqn:sol1_3}
          \end{equation}
          for some $\tilde{c}$. Using the same integrating factor as in part (a) (and recognizing that integrating factor is $\frac{1}{v_0}$, we see
          \begin{align*}
            \frac{\partial}{\partial y} \left( \frac{1}{v_0} u_1 \right) &= \frac{1}{v_0} \left( - v_0 \frac{\partial \widehat{u}}{\partial x} +
            \frac{\tilde{c}(x,t)}{D} \right) \\
            &= - \frac{\partial \widehat{u}}{\partial x} + \frac{1}{v_0} \frac{\tilde{c}(x,t)}{D}
          \end{align*}

          Integrating over $[0,1]$ as in part (a), we find
          \begin{align*}
            0 &= \left[ \frac{1}{v_0} u_1 \right]_0^1 \quad \parbox{5cm}{by periodicity as in (a)} \\
            &= \int_{0}^{1} \frac{\partial}{\partial y} \left( \frac{1}{v_0} u_1 \right) dy \\
            &= \int_{0}^{1} - \frac{\partial \widehat{u}}{\partial x} + \alpha(y) \frac{\tilde{c}(x,t)}{D} dy \\
            &= -\frac{\partial \widehat{u}}{\partial x} + \frac{\tilde{c}(x,t)}{D} \la v_0^{-1} \ra
          \end{align*}
          where $\la f \ra = \int_{0}^{1} f dy$.

          We can solve this for $\tilde{c}$ to get
          \[ \tilde{c}(x,t) = D \la v_0^{-1} \ra^{-1} \frac{\partial \widehat{u}}{\partial x} .\]

          Substituting this into \eqref{eqn:sol1_3}, we get
          \begin{equation}
            \frac{\partial u_1}{\partial y} + \frac{b(y)}{D} = - v_0 \frac{\partial \widehat{u}}{\partial x} + \la v_0^{-1} \ra^{-1}
            \frac{\partial \widehat{u}}{\partial x}
            \label{eqn:sol1_4}
          \end{equation}

          Now we look at the $O(1)$ equation:
          \[ \frac{\partial u_0}{\partial t} = \frac{\partial}{\partial y} \left( b u_2 + D \frac{\partial u_2}{\partial y} \right) +
          \frac{\partial}{\partial x} \left( D \frac{\partial u_0}{\partial x} + b u_1 + 2D \frac{\partial u_1}{\partial y} \right) .\]

          By using \eqref{eqn:sol1_4}, this gives us
          \begin{align*}
            \frac{\partial}{\partial y} \left( b u_2 + D \frac{\partial u_2}{\partial y} \right) &= \frac{\partial u_0}{\partial t} - \frac{\partial}{\partial
            x} \left( D \frac{\partial u_0}{\partial x} - D v_0 \frac{\partial \widehat{u}}{\partial x} + D \la v_0^{-1} \ra^{-1}
            \frac{\partial \widehat{u}}{\partial x} + \frac{\partial u_1}{\partial y} \right) \\
            &= v_0 \frac{\partial \widehat{u}}{\partial t} - \frac{\partial}{\partial x} \left( D \frac{\partial \widehat{u}}{\partial x} v_0 - D v_0
            \frac{\partial \widehat{u}}{\partial x} + D \la v_0^{-1} \ra^{-1} \frac{\partial \widehat{u}}{\partial x} + D \frac{\partial u_1}{\partial
            y} \right) \\
            &= v_0 \frac{\partial \widehat{u}}{\partial t} - \frac{\partial}{\partial x} \left( D \la v_0^{-1} \ra^{-1} \frac{\partial \widehat{u}}{\partial x} +
            D \frac{\partial u_1}{\partial y} \right)
          \end{align*}

          If we integrate this from 0 to 1 and use periodicity again, we find
          \begin{align*}
            0 &= \int_{0}^{1} \frac{\partial}{\partial y} \left( b u_2 + D \frac{\partial u_2}{\partial y} \right) dy \\
            &= \int_{0}^{1} \frac{\partial \widehat{u}}{\partial t} v_0 - \frac{\partial}{\partial x} \left( D \la v_0^{-1} \ra^{-1}
            \frac{\partial \widehat{u}}{\partial x} + \frac{\partial u_1}{\partial y} \right) dy \\
            &= \frac{\partial \widehat{u}}{\partial t} \la v_0 \ra - D \la v_0^{-1} \ra ^{-1} \frac{\partial^2 \widehat{u}}{\partial x^2} +
            \left. \frac{\partial u_1}{\partial x} \right|_0^1 \\
            &= \frac{\partial \widehat{u}}{\partial t} \la v_0 \ra - D \la v_0^{-1} \ra^{-1} \frac{\partial^2 \widehat{u}}{\partial x^2}
          \end{align*}

          Therefore,
          \[ \frac{\partial \widehat{u}}{\partial t} = \widehat{D} \frac{\partial^2 \widehat{u}}{\partial x^2} \]
          where $\widehat{D} = D \la v_0 \ra^{-1} \la v_0^{-1} \ra^{-1}$. Using Cauchy-Schwarz, we get
          \begin{align*}
            1 &= \left( \int_{0}^{1} \sqrt{v_0} \frac{1}{\sqrt{v_0}} dy \right)^2 \\
            &\leq \int_{0}^{1} v_0 dy \int_{0}^{1} \frac{1}{v_0} dy \\
            &= \la v_0 \ra \la v_0^{-1} \ra
          \end{align*}

          Thus, $\widehat{D} \leq D$.

      \end{enumerate}

    \end{solution}

    \pagebreak

  \item
    \begin{problem}
      Consider the system:
      \begin{align}
        \varepsilon \frac{\partial u}{\partial t} &= \varepsilon^2 \frac{\partial^2 u}{\partial x^2} + f(u,v), f(u,v) = u(1-u)(u-1-v),
        \label{eqn:prob2_1} \\
        \varepsilon \frac{\partial v}{\partial t} &= \frac{\partial^2 v}{\partial x^2} - f(u,v)
        \label{eqn:prob2_2}
      \end{align}
      posed on $0 < x < 1$ with boundary conditions:
      \begin{equation}
        \frac{\partial u}{\partial x} = \frac{\partial v}{\partial x} = 0 \text{ at } x = 0,1.
        \label{eqn:prob2_3}
      \end{equation}
      Values of $u$ and $v$ are assumed positive.

      \begin{enumerate}
        \item Show that, to leading order, $v$ is constant in $x$ (but may vary in time) and $u$ can only take values 0 or $1_v$.

        \item Given the above result, there must be one or more sharp transition layers within the domain, across which $u$ changes from 0 to $1+v$.
          Let there be one such front, and let $\varphi(t)$ be the position of this front; to leading order suppose
          \begin{equation}
            u = \begin{cases}
              1+v &\text{if } x< \varphi(t), \\
              0 &\text{if } x > \varphi(t).
            \end{cases}
            \label{eqn:prob2_4}
          \end{equation}
          Find an ordinary differential equation satisfied by this front position.
      \end{enumerate}

      In solving the above problem, it may be useful to use the following fact. Suppose we are given the equation:
      \begin{align}
        -c \frac{dw}{d \xi} &= \frac{\partial^2 w}{\partial \xi^2} + w(1-w)(w-b), b>1,
        \lim_{\xi \to - \infty} w(\xi) &= b, \lim_{\xi \to \infty} w(\xi) = 0.
        \label{eqn:prob2_10}
      \end{align}
      Then, there is a solution $w$ only when:
      \begin{equation}
        c = \frac{1}{\sqrt{2}} (b-2)
        \label{eqn:prob2_6}
      \end{equation}
      and $w$ is given by:
      \begin{equation}
        w = \frac{b}{2} \left( 1 + \tanh \left( \frac{b \xi}{ \sqrt{2} } + k \right) \right)
        \label{eqn:prob2_7}
      \end{equation}
      where $k$ is arbitrary.

    \end{problem}

    \begin{solution}
      \begin{enumerate}
        \item
          We expand
          \[ u = u_0 + \varepsilon u_1 + \dots \]
          and
          \[ v = v_0 + \varepsilon v_1 + \dots \]

          Then our $O(1)$ system is
          \[ \begin{cases}
              0 &= f(u_0,v_0) \\
              0 &= \frac{\partial^2 v_0}{\partial x_2} - f(u_0, v_0)
          \end{cases} \]
          where we have used a Taylor series expansion for $f$.
          By the first equation and the form of our nonlinearity, we know $u_0 = 0,1,1+v$. As in class, $1$ is an unstable fixed point for
          $\frac{\partial u}{\partial t} = f(u,v)$ because $v > 0$, so $u_0 = 0$ or $u_0 = 1+v$.

          By putting our first equation into our second, we get
          \[ \frac{\partial^2 v_0}{\partial x^2} = 0 .\]
          Solving this with the boundary conditions $\frac{\partial v_0}{\partial x} = 0$ at $x=0,1$ gives the solution
          \[ v_0 = c(t) \]
          for some function $c$ depending only on time.

        \item
          We introduce the boundary layer coordinates
          \[ x = \varphi(t) + \varepsilon \xi \]
          and the boundary layer solutions $U,V$ such that
          \[ U( \xi, t ) = u( \varphi(t) + \varepsilon \xi, t ), u(x,t) = U \left( \frac{x-\partial(t)}{\varepsilon}, t \right) \]
          and
          \[ V( \xi, t ) = v( \varphi(t) + \varepsilon \xi, t ), v(x,t) = V \left( \frac{x-\partial(t)}{\varepsilon}, t \right) \]

          Inserting these solutions into our original system gives the system
          \[ \begin{cases}
              - \varphi'(t) \frac{\partial U}{\partial \xi} + \varepsilon \frac{\partial U}{\partial t} &= \frac{\partial^2 U}{\partial \xi^2} +
              f(U,V) \\
              - \varphi'(t) \frac{\partial V}{\partial \xi} + \varepsilon \frac{\partial V}{\partial t} &= \frac{1}{\varepsilon^2}
              \frac{\partial^2 V}{\partial \xi^2} - f(U,V)
          \end{cases} \]

          First we verify that $V$ is (to leading order) remaining constant in the boundary layer (as we expect).

          By expanding $V$ asymptotically, we get the $O \left( \frac{1}{\varepsilon^2} \right)$ equation
          \begin{align*}
            &\frac{\partial^2 V_0}{\partial \xi^2} = 0 \\
            &\lim_{\xi \to -\infty} V_0 = c(t) \\
            &\lim_{\xi \to \infty} V_0 = c(t)
          \end{align*}
          The solution to this equation is $V_0 = c(t)$, and $V$ remains constant in our transition layer.

          By expanding $U$ asymptotically, we get the $O(1)$ equation
          \begin{align*}
            &0 = \frac{\partial^2 U_0}{\partial \xi^2} + \varphi'(t) \frac{\partial U_0}{\partial \xi} + f(U_0, V_0) \\
            &\lim_{\xi \to -\infty} U_0 = 1+V_0 \\
            &\lim_{\xi \to \infty} U_0 = 0
          \end{align*}

          Using our hint, this equation has a solution only when
          \[ \varphi'(t) = \frac{1}{\sqrt{2}} (V_0 - 1) = \frac{1}{\sqrt{2}} (v_0 - 1) , \]
          giving us the differential equation for the position of our transition layer.

      \end{enumerate}

    \end{solution}

    \pagebreak

  \item
    \begin{problem}
      Consider the following Cahn-Hilliard equation in $\Omega \subset \R^2$
      \begin{equation}
        \frac{\partial u}{\partial t} = - \Delta \left( \varepsilon \Delta u + \frac{1}{\varepsilon} f(u) \right),
        \label{eqn:prob3_1}
      \end{equation}
      where $f(u)$ is the usual cubic nonlinearity as in the Allen-Cahn equation. The stable roots are at $u=0$ and $u=1$, and we consider the
      balanced case:
      \begin{equation}
        \int_{0}^{1} f(s) ds = 0
        \label{eqn:prob3_2}
      \end{equation}

      Rewrite this equation as:
      \begin{align}
        \frac{\partial u}{\partial t} &= \Delta w \label{eqn:prob3_3} \\
        -w &= \varepsilon \Delta u + \frac{1}{\varepsilon} f(u).
        \label{eqn:prob3_4}
      \end{align}

      The boundary conditions imposed are:
      \begin{equation}
        \frac{\partial u}{\partial n} = \frac{\partial w}{\partial n} = 0 \text{ on } \partial \Omega
        \label{eqn:prob3_5}
      \end{equation}
      Thus, no-flux boundary conditions are imposed on both $u$ and $w$.

      \begin{enumerate}
        \item Expand $u$ and $w$ as:
          \begin{equation}
            u = u_0 + \varepsilon u_1 + \dots, w = w_0 + \varepsilon w_1 + \dots
            \label{eqn:prob3_6}
          \end{equation}
          and find the equations satisfied by $u_0$ and $w_0$. You should find that there will be regions where either $u_0 = 0$ or $u_0 = 1$.

        \item
          Let $\Gamma$ be the interface separating the region where $u_0 = 0$ and $u_0 = 1$. Introduce a moving coordinate system that fits with
          $\Gamma$ and find the equations in the inner transition layers together with matching conditions.

        \item
          Show that $w_0$ satisfies the following \textit{Mullins Sekerka} problem:
          \begin{align}
            \Delta w_0 &= 0 \text{ on } \Omega \setminus \Gamma \label{eqn:prob3_7} \\
            w_0 &= -A \kappa, v = \left[ \frac{\partial w_0}{\partial \nu} \right] \text{ on } \Gamma
            \label{eqn:prob3_8}
          \end{align}
          where $A$ is a positive constant that only depends on the cubic nonlinearity $f$, $\kappa$ is the curvature of the curve $\Gamma$ and $`N$
          is the unit normal vector on $\Gamma$, and $v$ is the normal velocity of the curve $\Gamma$.

      \end{enumerate}

    \end{problem}

    \begin{solution}

      \begin{enumerate}
        \item
          Expanding $u$ and $w$, we get from the $O \left( \frac{1}{\varepsilon} \right)$ equation for $w$ that
          \[ f(u_0) = 0 .\]
          Therefore, $u_0 = 0, 1$ because of the form of $f$ and the instability of the zero at $u_0=\alpha$.

          The $O(1)$ equations we get are
          \begin{align}
            \frac{\partial u_0}{\partial t} &= \Delta w_0 \\
            - w_0 &= f'(u_0) u_1
            \label{eqn:sol3_1}
          \end{align}

        \item
          Going through the same coordinate transformations done for the Allen-Cahn equation in class, we are able to express
          \[ \frac{\partial u}{\partial t} = \frac{\partial \widehat{u}}{\partial t} - v \frac{\partial \widehat{u}}{\partial r} +
          \frac{r}{1 + \kappa r} \frac{\partial v}{\partial s} \frac{\partial \widehat{u}}{\partial s} \]
          and
          \[ \Delta u = \frac{1}{1 + \kappa r} \left( \frac{\partial}{\partial r} \left( (1 + \kappa r) \frac{\partial \widehat{u}}{\partial r}
            \right) + \frac{\partial}{\partial s} \left( \frac{1}{1 + \kappa r} \frac{\partial \widehat{u}}{\partial s} \right) \right) \]
          and similarly for $w$.

          Now we can rewrite our equations as
          \begin{align*}
            &\frac{\partial \widehat{u}}{\partial t} - v \frac{\partial \widehat{u}}{\partial r} +
            \frac{r}{1 + \kappa r} \frac{\partial v}{\partial s} \frac{\partial \widehat{u}}{\partial s} \\
            &\quad = \frac{1}{1 + \kappa r} \left( \frac{\partial}{\partial r} \left( (1 + \kappa r) \frac{\partial \widehat{w}}{\partial r} \right) +
            \frac{\partial}{\partial s} \left( \frac{1}{1 + \kappa r} \frac{\partial \widehat{w}}{\partial s} \right) \right) \\
            &-\widehat{w} = \varepsilon \left( \frac{1}{1 + \kappa r} \left( \frac{\partial}{\partial r} \left( (1 + \kappa r) \frac{\partial
              \widehat{u}}{\partial r} \right) + \frac{\partial}{\partial s} \left( \frac{1}{1 + \kappa r} \frac{\partial \widehat{u}}{\partial s} \right) \right) \right) + \frac{1}{\varepsilon} f ( \widehat{u} )
          \end{align*}

          We introduce the inner layer coordinate
          \[ r = \varepsilon \rho \]
          and the inner layer solutions $U$ and $W$ satisfying
          \[ U(\rho,s,t) = \widehat{u} (\varepsilon \rho, s, t), \widehat{u}(r,s,t) = U \left( \frac{\rho}{\varepsilon}, s, t \right) \]
          and
          \[ W(\rho,s,t) = \widehat{w} (\varepsilon \rho, s, t), \widehat{w}(r,s,t) = W \left( \frac{\rho}{\varepsilon}, s, t \right) \]

          Plugging these solutions into our equations, we get
          \begin{align*}
            &\frac{\partial U}{\partial t} - \frac{1}{\varepsilon} v \frac{\partial U}{\partial \rho} +
            \frac{\varepsilon \rho}{1 + \varepsilon \kappa \rho} \frac{\partial v}{\partial s} \frac{\partial U }{\partial s} \\
            &\quad = \frac{1}{1 + \varepsilon \kappa \rho} \left( \frac{1}{\varepsilon} \frac{\partial}{\partial \rho} \left( (1 + \varepsilon \kappa \rho)
            \frac{1}{\varepsilon} \frac{\partial W}{\partial \rho} \right) +
            \frac{\partial}{\partial s} \left( \frac{1}{1 + \varepsilon \kappa \rho} \frac{\partial W}{\partial s} \right) \right) \\
            &-W = \varepsilon \left( \frac{1}{1 + \varepsilon \kappa \rho} \left( \frac{1}{\varepsilon} \frac{\partial}{\partial \rho} \left( (1 +
            \varepsilon \kappa \rho) \frac{1}{\varepsilon} \frac{\partial U}{\partial \rho} \right) +
            \frac{\partial}{\partial s} \left( \frac{1}{1 + \varepsilon \kappa \rho} \frac{\partial U}{\partial s} \right) \right) \right) + \frac{1}{\varepsilon} f ( U )
          \end{align*}

          We can use Taylor series to write
          \[ \frac{1}{1 + \varepsilon \kappa \rho} = 1 - \varepsilon \kappa \rho + \dots \]

          Using this Taylor series and expanding some terms, we can write our inner layer equations as
          \begin{align*}
            &\frac{\partial U}{\partial t} - \frac{1}{\varepsilon} v \frac{\partial U}{\partial \rho} +
            \varepsilon \rho ( 1 - \varepsilon \kappa \rho ) \frac{\partial v}{\partial s} \frac{\partial U }{\partial s} \\
            &\quad = (1 - \varepsilon \kappa \rho + \dots ) \left( \left( \frac{1}{\varepsilon} \kappa \frac{\partial W}{\partial \rho} + \frac{1}{\varepsilon^2} (1 +
            \varepsilon \kappa \rho ) \frac{\partial^2 W}{\partial \rho^2} \right) + \frac{\partial}{\partial s} \left( ( 1 - \varepsilon \kappa \rho +
            \dots ) \frac{\partial W}{\partial s} \right) \right) \\
            -W &= \varepsilon \left( (1 - \varepsilon \kappa \rho) \left( \left( \frac{1}{\varepsilon} \kappa \frac{\partial U}{\partial \rho} +
            \frac{1}{\varepsilon^2} (1+ \varepsilon \kappa \rho) \frac{\partial^2 U}{\partial \rho^2} \right) + \frac{\partial}{\partial s} \left( (1
            - \varepsilon \kappa \rho ) \frac{\partial U}{\partial s} \right) \right) \right) + \frac{1}{\varepsilon} f(U)
          \end{align*}

          Now we expand $U = U_0 + \varepsilon U_1 + \dots$ and $W = W_0 + \varepsilon W_1 + \dots$.
          With this, we only get a $O \left( \frac{1}{\varepsilon^2} \right)$ contribution from the first equation, which is
          \[ 0 = \frac{\partial^2 W_0}{\partial \rho^2} .\]

          For the $O \left( \frac{1}{\varepsilon} \right)$ equations, we get
          \begin{align}
            -v \frac{\partial U_0}{\partial \rho} &= \frac{\partial^2 W_1}{\partial \rho^2} + \kappa \frac{\partial W_0}{\partial \rho} \nonumber \\
            0 &= \frac{\partial^2 U_0}{\partial \rho^2} + f(U_0)
            \label{eqn:sol3_2}
          \end{align}

          We also get the $O(1)$ equation
          \begin{equation*}
            - W_0 = \frac{\partial^2 U_1}{\partial \rho^2} + \kappa \frac{\partial U_0}{\partial \rho} + f'(U_0) U_1
          \end{equation*}
          from the second equation of our system.

          We also have the matching conditions
          \begin{align*}
            \lim_{\rho \to -\infty} U_0(\rho) &= 1 \\
            \lim_{\rho \to \infty} U_0(\rho) &= 0 \\
            \lim_{\rho \to \infty} W_0(\rho) &= w_- \\
            \lim_{\rho \to \infty} W_0(\rho) &= w_+
          \end{align*}

        \item
          By our results from (a), we know $u_0$ is constant away from $\Gamma$. Therefore, by \eqref{eqn:sol3_1},
          \[ \Delta w_0 = \frac{\partial u_0}{\partial t} = 0 \]
          away from $\Gamma$.

          From the $O(1)$ equation we have included, we have
          \[ \frac{\partial^2 U_1}{\partial \rho^2} + f'(U_0) U_1 = -W_0 - \kappa \frac{\partial U_0}{\partial \rho} \]

          Define the operator $L$ defined by
          \[ L \varphi = \frac{\partial^2 \varphi}{\partial \rho^2} + f'(U_0) U_1 \]
          We notice that $L$ is self-adjoint.

          In order for our $O(1)$ equation above to have a solution for $U_1$, we need $-W_0 - \kappa \frac{\partial U_0}{\partial \rho} \in (\ker
          L^\ast)^\perp = (\ker L)^\perp$.
          By taking the derivative of \eqref{eqn:sol3_2}, we see
          \[ \frac{\partial^2}{\partial \rho^2} \left( \frac{\partial U_0}{\partial \rho} \right) + f'(\rho) \frac{\partial U_0}{\partial \rho} = 0 \]
          Therefore, $\frac{\partial U_0}{\partial \rho} \in \ker(L)$. So we need,
          \[ 0 = \int_{-\infty}^{\infty} \left( -W_0 - \kappa \frac{\partial U_0}{\partial \rho} \right) \frac{\partial U_0}{\partial \rho} d\rho \]

          If we assume $W_0$ is constant, we get
          \begin{align*}
            W_0 &= -W_0 \int_{-\infty}^{\infty} \frac{\partial U_0}{\partial \rho} d\rho \\
            &= \kappa \int_{-\infty}^{\infty} \left( \frac{\partial^2 U_0}{\partial \rho^2} \right)^2 d\rho \\
            &= A \kappa
          \end{align*}
          as desired.

          Our assumption that $W_0$ is constant is in fact true. We have the $O \left( \frac{1}{\varepsilon^2} \right)$ equation
          \[ 0 = \frac{\partial^2 W_0}{\partial \rho^2} \]
          which has solutions $W_0 = c(s,t) \rho + d(s,t)$. From our matching conditions, we must have $W_0$ taking on finite values at $-\infty$ and
          $\infty$. The only way this can happen is if $c = 0$ and $w_-=w_+=d(s,t)$.

          With this, we have from \eqref{eqn:sol3_2},
          \[ -v \frac{\partial U_0}{\partial \rho} = \frac{\partial^2 W_1}{\partial \rho^2} .\]

          Integrating from $-\infty$ to $\infty$ in $\rho$ and using the fact that $v$ only depends on $s$ and $t$, we have
          \begin{align*}
            v &= -v \left( \lim_{\rho \to \infty} U_0(\rho) - \lim_{\rho \to -\infty} U_0(\rho) \right) \\
            &= -v \int_{-\infty}^{\infty} \frac{\partial U_0}{\partial \rho} d\rho \\
            &= \int_{-\infty}^{\infty} \frac{\partial^2 W_1}{\partial \rho^2} d\rho \\
            &= \lim_{\rho \to \infty} \frac{\partial W_1}{\partial \rho} (\rho) - \lim_{\rho \to -\infty} \frac{\partial W_1}{\partial \rho}(\rho)
          \end{align*}

          By using the matching conditions that $W_1$ matches $w_0$ and the fact that $\rho$ is our coordinate that is tangent to $\Gamma$ (so it
          points in the direction of $\nu$) this tells us
          \[ v = \left[ \frac{\partial w_0}{\partial \nu} \right] .\]

      \end{enumerate}

    \end{solution}

    \pagebreak

  \item
      \begin{problem}
      Consider a fluid placed between two plates that are a distance $\varepsilon L$ apart. In terms of coordinates, the two plates correspond to $z=0$
      and $z = \varepsilon L(x,y)$, where $L(x,y)$ is some function of $x$ and $y$. The equation satisfied by the fluid is:
      \begin{align}
        \mu \Delta \mathbf{u} = \nabla p, \nabla \cdot \mathbf{u} &= 0 \text{ for } 0 < z < \varepsilon L \label{eqn:prob4_1} \\
        \mathbf{u} &= 0 \text{ on } z = 0, \varepsilon L
        \label{eqn:prob4_2}
      \end{align}
      We consider the limit when $\varepsilon$ is small. Such a flow is called \textit{Hele-Shaw flow}.

      \begin{enumerate}
        \item Let $\mathbf{u} = (u,v,w)^T$. Rescale the equations so that $z = \varepsilon Z.$

        \item Expand $u,v,w,p$ in powers of $\varepsilon$ (let $u = u_0 + \varepsilon u_1 + \varepsilon^2 u_2 + \dots$ and so on). Show that $u_0, u_1,
          v_0, v_1, w_0, w_1, w_2$ are all 0. Show that $p_0$ does not depend on $Z$. (This shows that the leading order contribution to the velocity
          is oder $\varepsilon^2$).

        \item Find the equations satisfied by $p_0, u_2, v_2$.

      \end{enumerate}

    \end{problem}

    \begin{solution}

      \begin{enumerate}
        \item
          Letting $z = \varepsilon Z$, we have $\frac{\partial}{\partial z} = \frac{1}{\varepsilon} \frac{\partial}{\partial Z}$. So our equations
          become
          \begin{align*}
            \mu \begin{pmatrix}
              \partial_{xx} u + \partial_{yy} u + \frac{1}{\varepsilon^2} \partial_{ZZ} u \\[6pt]
              \partial_{xx} v + \partial_{yy} v + \frac{1}{\varepsilon^2} \partial_{ZZ} v \\[6pt]
              \partial_{xx} w + \partial_{yy} w + \frac{1}{\varepsilon^2} \partial_{ZZ} w
            \end{pmatrix}
            &= \begin{pmatrix}
              \partial_x p \\[6pt]
              \partial_y p \\[6pt]
              \frac{1}{\varepsilon} \partial_Z p
            \end{pmatrix}
            \text{ for } 0 < Z < L \\
            \partial_x u + \partial_y v + \frac{1}{\varepsilon} \partial_Z w &= 0 \text{ for } 0 < Z < L \\
            u = v = w &= 0 \text{ on } Z = 0,L
          \end{align*}

        \item

          Let $u = u_0 + \varepsilon u_1 + \dots$ and similarly for $v,w,$ and $p$.
          This gives the following $O \left( \frac{1}{\varepsilon^2} \right)$ equation:
          \[ \mu \begin{pmatrix}
            \partial_{ZZ} u_0 \\[6pt]
            \partial_{ZZ} v_0 \\[6pt]
            \partial_{ZZ} w_0 \\[6pt]
            \end{pmatrix}
            =
            \begin{pmatrix}
              0 \\[6pt]
              0 \\[6pt]
              0
            \end{pmatrix}
          \]
          with the boundary conditions $u_0 = v_0 = w_0 = 0$ at $Z=0,L$.
          By solving the first of these equations, we get $u_0 = c(x,y) Z + d(x,y)$. With our boundary conditions, we get $u_0 = 0$. Similarly, $v_0$
          and $w_0$ are both zero.

          Next, we get the $O \left( \frac{1}{\varepsilon} \right)$ equations:
          \begin{align*}
            \mu \begin{pmatrix}
              \partial_{ZZ} u_1 \\[6pt]
              \partial_{ZZ} v_1 \\[6pt]
              \partial_{ZZ} w_1
            \end{pmatrix}
            &=
            \begin{pmatrix}
              0 \\[6pt]
              0 \\[6pt]
              \partial_Z p_0
            \end{pmatrix} \quad \text{for } 0 < Z < L \\
            \partial_Z w_0 &= 0 \quad \text{for } 0 < Z < L
          \end{align*}
          As in the $O \left( \frac{1}{\varepsilon^2} \right)$ case, we get $u_1 = v_1 = 0$. Taking an antiderivative of our last equation, we get
          \begin{equation}
            \mu \partial_Z w_1 = p_0 + c(x,y)
            \label{eqn:sol4_1}
          \end{equation}
          for some $c$ and all $0<Z<L$.

          Moving on to the $O(1)$ equation, we have
          \begin{align*}
            \mu \begin{pmatrix}
              \partial_{ZZ} u_2 \\[6pt]
              \partial_{ZZ} v_2 \\[6pt]
              \partial_{ZZ} w_2
            \end{pmatrix}
            &=
            \begin{pmatrix}
              \partial_x p_0 \\[6pt]
              \partial_y p_0 \\[6pt]
              \partial_Z p_0
            \end{pmatrix}
            \quad \text{for } 0 < Z < L \\
            \partial_Z w_1 &= 0 \quad \text{for } 0 < Z < L
          \end{align*}
          where we have used the fact that $u_0 = v_0 = w_0 = 0$.

          Because of the zero boundary conditions and the fact that $\partial_Z w_1 = 0$, we know $w_1 = 0$. We also know from \eqref{eqn:sol4_1} that
          \begin{align*}
            0 &= \partial_Z w_1 \\
            &= \frac{1}{\mu} ( p_0 + c(x,y) )
          \end{align*}
          Thus, $p_0$ is independent of $Z$.

          We have the following $O(\varepsilon)$ equation:
          \begin{align*}
            \mu \begin{pmatrix}
              \partial_{ZZ} u_3 \\[6pt]
              \partial_{ZZ} v_3 \\[6pt]
              \partial_{ZZ} w_3
            \end{pmatrix}
            &=
            \begin{pmatrix}
              \partial_x p_1 \\[6pt]
              \partial_y p_1 \\[6pt]
              \partial_Z p_2
            \end{pmatrix}
            \quad \text{for } 0 < Z < L \\
            \partial_Z w_2 &= 0 \quad \text{for } 0 < Z < L
          \end{align*}

          Once again by using the boundary conditions and $\partial_Z w_2 = 0$, we get $w_2 = 0$.

        \item
          We see from the $O(1)$ equations that $p_0, u_2,$ and $v_2$ satisfy
          \begin{align*}
            \mu \partial_{ZZ} u_2 &= \partial_x p_0 \\
            \mu \partial_{ZZ} v_2 &= \partial_y p_0
          \end{align*}

          Using the fact that we know $p_0$ is independent of $Z$, we know
          \begin{align*}
            \mu \partial_{ZZ} u_2 &= - \partial_x c(x,y) \\
            \mu \partial_{ZZ} v_2 &= - \partial_y c(x,y)
          \end{align*}


      \end{enumerate}

    \end{solution}

  \end{enumerate}
\end{document}


