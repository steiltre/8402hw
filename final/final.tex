%        File: final.tex
%     Created: Thu May 05 09:00 PM 2016 C
% Last Change: Thu May 05 09:00 PM 2016 C
%

\documentclass[a4paper]{article}

\title{Math 8402 Take-Home Final }
\date{5/13/16}
\author{Trevor Steil}

\usepackage{amsmath}
\usepackage{amsthm}
\usepackage{amssymb}
\usepackage{esint}

\newtheorem{theorem}{Theorem}[section]
\newtheorem{corollary}{Corollary}[section]
\newtheorem{proposition}{Proposition}[section]
\newtheorem{lemma}{Lemma}[section]
\newtheorem*{claim}{Claim}
\newtheorem*{problem}{Problem}
%\newtheorem*{lemma}{Lemma}
\newtheorem{definition}{Definition}[section]

\newcommand{\R}{\mathbb{R}}
\newcommand{\N}{\mathbb{N}}
\newcommand{\C}{\mathbb{C}}
\newcommand{\Z}{\mathbb{Z}}
\newcommand{\supp}[1]{\mathop{\mathrm{supp}}\left(#1\right)}
\newcommand{\lip}[1]{\mathop{\mathrm{Lip}}\left(#1\right)}
\newcommand{\curl}{\mathrm{curl}}
\newcommand{\la}{\left \langle}
\newcommand{\ra}{\right \rangle}
\renewcommand{\vec}[1]{\mathbf{#1}}

\newenvironment{solution}{\emph{Solution.}}

\begin{document}
\maketitle
\begin{enumerate}
  \item
    \begin{problem}
      Consider the following equation:
      \begin{equation}
        \frac{\partial u}{\partial t} = \frac{\partial}{\partial x} \left( \frac{1}{\varepsilon} b \left( \frac{x}{\varepsilon} \right) u + D
        \frac{\partial u}{\partial x} \right)
        \label{eqn:prob1_1}
      \end{equation}
      where $D>0$ and $b(y)$ is a periodic function in $y$ with period 1 whose average is equal to 0:
      \begin{equation}
        \int_{0}^{1} b(y) dy = 0
        \label{eqn:prob1_2}
      \end{equation}

      We will be interested in positive soltuions ($u>0$). Assume that $u(x,t)$ can be written as:
      \begin{equation}
        u(x,t) = u(x,y,t) = u_0(x,y,t) + \varepsilon u_1(x,y,t) + \varepsilon^2 u_2(x,y,t) + \dots
        \label{eqn:prob1_3}
      \end{equation}
      where $y = \frac{x}{\varepsilon} $.

      \begin{enumerate}
        \item Show that $u_0(x,y,t)$ can be written as:
          \begin{equation}
            u_0(x,y,t) = \widehat{u} (x,t) v_0(y)
            \label{eqn:prob1_4}
          \end{equation}
          where $v_0(y)$ is a function that does not depend on $x$ or $t$.

        \item Show that $\widehat{u}(x,t)$ satisfies the equation:
          \begin{equation}
            \frac{\partial \widehat{u}}{\partial t} = \widehat{D} \frac{\partial^2 \widehat{u}}{\partial x^2} .
            \label{eqn:prob1_5}
          \end{equation}
          Obtain an explicit expression for $\widehat{D}$ and show that $\widehat{D}$ is never greater than $D$.

      \end{enumerate}

    \end{problem}

    \begin{solution}

      \begin{enumerate}
        \item
          Because $y = \frac{x}{\varepsilon}$, we write our equation as
          \[ \frac{\partial u}{\partial t} = \left( \frac{\partial}{\partial x} + \frac{1}{\varepsilon} \right) \left( \frac{1}{\varepsilon} b(y) u +
          D \left( \frac{\partial}{\partial x} + \frac{1}{\varepsilon} \frac{\partial}{\partial y} \right) u \right) .\]

          By expanding $u$ asymptotically, we get the $O \left( \frac{1}{\varepsilon^2} \right)$ equation
          \begin{align*}
            0 &= \frac{\partial}{\partial y} ( b(y) u_0 ) + \frac{\partial}{\partial y} \left( D \frac{\partial}{\partial y} u_0 \right) \\
            &= \frac{\partial}{\partial y} \left( b(y) u_0 + D \frac{\partial}{\partial y} u_0 \right)
          \end{align*}
          Therefore,
          \[ b(y) u_0 + D \frac{\partial}{\partial y} u_0 = \widehat{u} (x,t) \]
          for some function $\widehat{u}$.

          Then finding an integrating factor gives
          \[ \frac{\partial}{\partial y} \left( u_0 \exp \left( \frac{1}{D} \int_{0}^{y} b(z) dz \right) \right) = \frac{\widehat{u}}{D} .\]
          Therefore,
          \[ u_0 = \widehat{u}(x,t) v_0(y) \]
          where
          \[ v_0(y) = y \exp \left( \frac{1}{D} \int_{0}^{y} b(z) dz \right) .\]

        \item

      \end{enumerate}

    \end{solution}

  \item
    \begin{problem}
      Consider the system:
      \begin{align}
        \varepsilon \frac{\partial u}{\partial t} &= \varepsilon^2 \frac{\partial^2 u}{\partial x^2} + f(u,v), f(u,v) = u(1-u)(u-1-v),
        \label{eqn:prob2_1} \\
        \varepsilon \frac{\partial v}{\partial t} &= \frac{\partial^2 v}{\partial x^2} - f(u,v)
        \label{eqn:prob2_2}
      \end{align}
      posed on $0 < x < 1$ with boundary conditions:
      \begin{equation}
        \frac{\partial u}{\partial x} = \frac{\partial v}{\partial x} = 0 \text{ at } x = 0,1.
        \label{eqn:prob2_3}
      \end{equation}
      Values of $u$ and $v$ are assumed positive.

      \begin{enumerate}
        \item Show that, to leading order, $v$ is constant in $x$ (but may vary in time) and $u$ can only take values 0 or $1_v$.

        \item Given the above result, there must be one or more sharp transition layers within the domain, across which $u$ changes from 0 to $1+v$.
          Let there be one such front, and let $\varphi(t)$ be the position of this front; to leading order suppose
          \begin{equation}
            u = \begin{cases}
              1+v &\text{if } x< \varphi(t), \\
              0 &\text{if } x > \varphi(t).
            \end{cases}
            \label{eqn:prob2_4}
          \end{equation}
          Find an ordinary differential equation satisfied by this front position.
      \end{enumerate}

      In solving the above problem, it may be useful to use the following fact. Suppose we are given the equation:
      \begin{align}
        -c \frac{dw}{d \xi} &= \frac{\partial^2 w}{\partial \xi^2} + w(1-w)(w-b), b>1,
        \lim_{\xi \to - \infty} w(\xi) &= b, \lim_{\xi \to \infty} w(\xi) = 0.
        \label{eqn:prob2_10}
      \end{align}
      Then, there is a solution $w$ only when:
      \begin{equation}
        c = \frac{1}{\sqrt{2}} (b-2)
        \label{eqn:prob2_6}
      \end{equation}
      and $w$ is given by:
      \begin{equation}
        w = \frac{b}{2} \left( 1 + \tanh \left( \frac{b \xi}{ \sqrt{2} } + k \right) \right)
        \label{eqn:prob2_7}
      \end{equation}
      where $k$ is arbitrary.

    \end{problem}

    \begin{solution}
      \begin{enumerate}
        \item
          We expand
          \[ u = u_0 + \varepsilon u_1 + \dots \]
          and
          \[ v = v_0 + \varepsilon v_1 + \dots \]

          Then our $O(1)$ system is
          \[ \begin{cases}
              0 &= f(u_0,v_0) \\
              0 &= \frac{\partial^2 v_0}{\partial x_2} - f(u_0, v_0)
          \end{cases} \]
          where we have used a Taylor series expansion for $f$.
          By the first equation and the form of our nonlinearity, we know $u_0 = 0,1,1+v$. As in class, $1$ is an unstable fixed point for
          $\frac{\partial u}{\partial t} = f(u,v)$ because $v > 0$, so $u_0 = 0$ or $u_0 = 1+v$.

          By putting our first equation into our second, we get
          \[ \frac{\partial^2 v_0}{\partial x^2} = 0 .\]
          Solving this with the boundary conditions $\frac{\partial v_0}{\partial x} = 0$ at $x=0,1$ gives the solution
          \[ v_0 = c(t) \]
          for some function $c$ depending only on time.

        \item
          We introduce the boundary layer coordinates
          \[ x = \varphi(t) + \varepsilon \xi \]
          and the boundary layer solutions $U,V$ such that
          \[ U( \xi, t ) = u( \varphi(t) + \varepsilon \xi, t ), u(x,t) = U \left( \frac{x-\partial(t)}{\varepsilon}, t \right) \]
          and
          \[ V( \xi, t ) = v( \varphi(t) + \varepsilon \xi, t ), v(x,t) = V \left( \frac{x-\partial(t)}{\varepsilon}, t \right) \]

          Inserting these solutions into our original system gives the system
          \[ \begin{cases}
              - \varphi'(t) \frac{\partial U}{\partial \xi} + \varepsilon \frac{\partial U}{\partial t} &= \frac{\partial^2 U}{\partial \xi^2} +
              f(U,V) \\
              - \varphi'(t) \frac{\partial V}{\partial \xi} + \varepsilon \frac{\partial V}{\partial t} &= \frac{1}{\varepsilon^2}
              \frac{\partial^2 V}{\partial \xi^2} - f(U,V)
          \end{cases} \]

          First we verify that $V$ is (to leading order) remaining constant in the boundary layer (as we expect).

          By expanding $V$ asymptotically, we get the $O \left( \frac{1}{\varepsilon^2} \right)$ equation
          \begin{align*}
            &\frac{\partial^2 V_0}{\partial \xi^2} = 0 \\
            &\lim_{\xi \to -\infty} V_0 = c(t) \\
            &\lim_{\xi \to \infty} V_0 = c(t)
          \end{align*}
          The solution to this equation is $V_0 = c(t)$, and $V$ remains constant in our transition layer.

          By expanding $U$ asymptotically, we get the $O(1)$ equation
          \begin{align*}
            &0 = \frac{\partial^2 U_0}{\partial \xi^2} + \varphi'(t) \frac{\partial U_0}{\partial \xi} + f(U_0, V_0) \\
            &\lim_{\xi \to -\infty} U_0 = 1+V_0 \\
            &\lim_{\xi \to \infty} U_0 = 0
          \end{align*}

          Using our hint, this equation has a solution only when
          \[ \varphi'(t) = \frac{1}{\sqrt{2}} (V_0 - 1) = \frac{1}{\sqrt{2}} (v_0 - 1) , \]
          giving us the differential equation for the position of our transition layer.

      \end{enumerate}

    \end{solution}

  \item
    \begin{problem}
      Consider the following Cahn-Hilliard equation in $\Omega \subset \R^2$
      \begin{equation}
        \frac{\partial u}{\partial t} = - \Delta \left( \varepsilon \Delta u + \frac{1}{\varepsilon} f(u) \right),
        \label{eqn:prob3_1}
      \end{equation}
      where $f(u)$ is the usual cubic nonlinearity as in the Allen-Cahn equation. The stable roots are at $u=0$ and $u=1$, and we consider the
      balanced case:
      \begin{equation}
        \int_{0}^{1} f(s) ds = 0
        \label{eqn:prob3_2}
      \end{equation}

      Rewrite this equation as:
      \begin{align}
        \frac{\partial u}{\partial t} &= \Delta w \label{eqn:prob3_3} \\
        -w &= \varepsilon \Delta u + \frac{1}{\varepsilon} f(u).
        \label{eqn:prob3_4}
      \end{align}

      The boundary conditions imposed are:
      \begin{equation}
        \frac{\partial u}{\partial n} = \frac{\partial w}{\partial n} = 0 \text{ on } \partial \Omega
        \label{eqn:prob3_5}
      \end{equation}
      Thus, no-flux boundary conditions are imposed on both $u$ and $w$.

      \begin{enumerate}
        \item Expand $u$ and $w$ as:
          \begin{equation}
            u = u_0 + \varepsilon u_1 + \dots, w = w_0 + \varepsilon w_1 + \dots
            \label{eqn:prob3_6}
          \end{equation}
          and find the equations satisfied by $u_0$ and $w_0$. You should find that there will be regions where either $u_0 = 0$ or $u_0 = 1$.

        \item
          Let $\Gamma$ be the interface separating the region where $u_0 = 0$ and $u_0 = 1$. Introduce a moving coordinate system that fits with
          $\Gamma$ and find the equations in the inner transition layers together with matching conditions.

        \item
          Show that $w_0$ satisfies the following \textit{Mullins Sekerka} problem:
          \begin{align}
            \Delta w_0 &= 0 \text{ on } \Omega \setminus \Gamma \label{eqn:prob3_7} \\
            w_0 &= -A \kappa, v = \left[ \frac{\partial w_0}{\partial \nu} \right] \text{ on } \Gamma
            \label{eqn:prob3_8}
          \end{align}
          where $A$ is a positive constant that only depends on the cubic nonlinearity $f$, $\kappa$ is the curvature of the curve $\Gamma$ and $`N$
          is the unit normal vector on $\Gamma$, and $v$ is the normal velocity of the curve $\Gamma$.

      \end{enumerate}

    \end{problem}

    \begin{solution}

      \begin{enumerate}
        \item
          Expanding $u$ and $w$, we get from the $O \left( \frac{1}{\varepsilon} \right)$ equation for $w$ that
          \[ f(u_0) = 0 .\]
          Therefore, $u_0 = 0, 1$ because of the form of $f$ and the instability of the zero at $u_0=\alpha$.

          The $O(1)$ equations we get are
          \begin{align*}
            \frac{\partial u_0}{\partial t} &= \Delta w_0 \\
            - w_0 &= f'(u_0) u_1
          \end{align*}

        \item
          Going through the same coordinate transformations done for the Allen-Cahn equation in class, we are able to express
          \[ \frac{\partial u}{\partial t} = \frac{\partial \widehat{u}}{\partial t} - v \frac{\partial \widehat{u}}{\partial r} +
          \frac{r}{1 + \kappa r} \frac{\partial v}{\partial s} \frac{\partial \widehat{u}}{\partial s} \]
          and
          \[ \Delta u = \frac{1}{1 + \kappa r} \left( \frac{\partial}{\partial r} (1 + \kappa r) \frac{\partial \widehat{u}}{\partial r} +
          \frac{\partial}{\partial s} \left( \frac{1}{1 + \kappa r} \frac{\partial \widehat{u}}{\partial s} \right) \right) \]
          and similarly for $w$.

          Now we can rewrite our equations as
          \begin{align*}
            &\frac{\partial \widehat{u}}{\partial t} - v \frac{\partial \widehat{u}}{\partial r} +
          \frac{r}{1 + \kappa r} \frac{\partial v}{\partial s} \frac{\partial \widehat{u}}{\partial s} \\
          &\quad = \frac{1}{1 + \kappa r} \left( \frac{\partial}{\partial r} (1 + \kappa r) \frac{\partial \widehat{w}}{\partial r} +
          \frac{\partial}{\partial s} \left( \frac{1}{1 + \kappa r} \frac{\partial \widehat{w}}{\partial s} \right) \right) \\
          &-\widehat{w} = \varepsilon \left( \frac{1}{1 + \kappa r} \left( \frac{\partial}{\partial r} (1 + \kappa r) \frac{\partial \widehat{u}}{\partial r} +
          \frac{\partial}{\partial s} \left( \frac{1}{1 + \kappa r} \frac{\partial \widehat{u}}{\partial s} \right) \right) \right) \\
          &\quad \quad + \frac{1}{\varepsilon} f ( \widehat{u} )
          \end{align*}

          We introduce the inner layer coordinate
          \[ r = \varepsilon \rho \]
          and the inner layer solutions $U$ and $W$ satisfying
          \[ U(\rho,s,t) = \widehat{u} (\varepsilon \rho, s, t), \widehat{u}(r,s,t) = U \left( \frac{\rho}{\varepsilon}, s, t \right) \]
          and
          \[ W(\rho,s,t) = \widehat{w} (\varepsilon \rho, s, t), \widehat{w}(r,s,t) = W \left( \frac{\rho}{\varepsilon}, s, t \right) \]

          Plugging these solutions into our equations, we get
          \begin{align*}
            &\frac{\partial U}{\partial t} - \frac{1}{\varepsilon} v \frac{\partial U}{\partial \rho} +
            \frac{\varepsilon \rho}{1 + \varepsilon \kappa \rho} \frac{\partial v}{\partial s} \frac{\partial U }{\partial s} \\
            &\quad = \frac{1}{1 + \varepsilon \kappa \rho} \left( \frac{1}{\varepsilon} \frac{\partial}{\partial \rho} (1 + \varepsilon \kappa \rho)
            \frac{1}{\varepsilon} \frac{\partial W}{\partial \rho} +
            \frac{\partial}{\partial s} \left( \frac{1}{1 + \varepsilon \kappa \rho} \frac{\partial W}{\partial s} \right) \right) \\
            &-W = \varepsilon \left( \frac{1}{1 + \varepsilon \kappa \rho} \left( \frac{1}{\varepsilon} \frac{\partial}{\partial \rho} (1 +
            \varepsilon \kappa \rho) \frac{\partial U}{\partial r} +
            \frac{\partial}{\partial s} \left( \frac{1}{1 + \varepsilon \kappa \rho} \frac{\partial U}{\partial s} \right) \right) \right) \\
            &\quad \quad + \frac{1}{\varepsilon} f ( U )
          \end{align*}

        \item

      \end{enumerate}

    \end{solution}

  \item
      \begin{problem}
      Consider a fluid placed between two plates that are a distance $\varepsilon L$ apart. In terms of coordinates, the two plates correspond to $z=0$
      and $z = \varepsilon L(x,y)$, where $L(x,y)$ is some function of $x$ and $y$. The equation satisfied by the fluid is:
      \begin{align}
        \mu \Delta \mathbf{u} = \nabla p, \nabla \cdot \mathbf{u} &= 0 \text{ for } 0 < z < \varepsilon L \label{eqn:prob4_1} \\
        \mathbf{u} &= 0 \text{ on } z = 0, \varepsilon L
        \label{eqn:prob4_2}
      \end{align}
      We consider the limit when $\varepsilon$ is small. Such a flow is called \textit{Hele-Shaw flow}.

      \begin{enumerate}
        \item Let $\mathbf{u} = (u,v,w)^T$. Rescale the equations so that $z = \varepsilon Z.$

        \item Expand $u,v,w,p$ in powers of $\varepsilon$ (let $u = u_0 + \varepsilon u_1 + \varepsilon^2 u_2 + \dots$ and so on). Show that $u_0, u_1,
          v_0, v_1, w_0, w_1, w_2$ are all 0. Show that $p_0$ does not depend on $Z$. (This shows that the leading order contribution to the velocity
          is oder $\varepsilon^2$).

        \item Find the equations satisfied by $p_0, u_2, v_2$.

      \end{enumerate}

    \end{problem}

    \begin{solution}

      \begin{enumerate}
        \item

        \item

        \item

      \end{enumerate}

    \end{solution}

  \end{enumerate}
\end{document}


