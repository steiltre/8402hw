%        File: hw3.tex
%     Created: Tue Mar 22 12:00 PM 2016 C
% Last Change: Tue Mar 22 12:00 PM 2016 C
%

\documentclass[a4paper]{article}

\title{Math 8402 Homework 3 }
\date{4/4/16}
\author{Trevor Steil}

\usepackage{amsmath}
\usepackage{amsthm}
\usepackage{amssymb}
\usepackage{bigints}

\newtheorem{theorem}{Theorem}[section]
\newtheorem{corollary}{Corollary}[section]
\newtheorem{proposition}{Proposition}[section]
\newtheorem{lemma}{Lemma}[section]
\newtheorem*{claim}{Claim}
\newtheorem*{problem}{Problem}
%\newtheorem*{lemma}{Lemma}
\newtheorem{definition}{Definition}[section]

\newcommand{\R}{\mathbb{R}}
\newcommand{\N}{\mathbb{N}}
\newcommand{\C}{\mathbb{C}}
\newcommand{\supp}[1]{\mathop{\mathrm{supp}}\left(#1\right)}
\newcommand{\lip}[1]{\mathop{\mathrm{Lip}}\left(#1\right)}
\newcommand{\curl}{\mathrm{curl}}
\newcommand{\la}{\left \langle}
\newcommand{\ra}{\right \rangle}
\renewcommand{\vec}[1]{\mathbf{#1}}

\newenvironment{solution}{\emph{Solution.}}

\begin{document}
\maketitle

\begin{enumerate}
  \item \begin{problem}
      Consider
      \[ y'' + \varepsilon f(y,y') + y = 0, t>0 \]
      where $\varepsilon$ is small and $f$ is some smooth function.
      \begin{enumerate}
        \item Use the multiple time scale expansion:
          \[ y = y_0(t_1, t_2) + \varepsilon y_1(t_1, t_2) + \dots ,
          t_1 = t, t_2 = \varepsilon t \]
          and show that $y_0$ can be written as:
          \[ y_0 = A(t_2) cos(\tau), \tau = t_1 + \phi(t_2) \]
          where $\phi(t_2)$ is some function of $t_2$.

        \item Show that the $O(\varepsilon)$ equation is
          \[ \frac{\partial^2 y_1}{\partial t_1^2} + y_1 = 2 \left(
            \frac{\partial A}{\partial t_2} \sin \tau +
            A \frac{\partial \phi}{\partial t_2} \cos \tau \right)
          - f \left( y_0, \frac{\partial y_0}{\partial \tau} \right) \]

        \item In order to avoid secular terms, show that the conditions to be
          satisfied are:
          \[ \frac{\partial A}{\partial t_2} = \frac{1}{2 \pi} \int_{0}^{2 \pi} f \left( y_0, \frac{\partial y_0}{\partial \tau} \right) \sin \tau d \tau \]
          and
          \[ A \frac{\partial \phi}{\partial t_2} = \frac{1}{2 \pi} \int_{0}^{2 \pi} f \left( y_0, \frac{\partial y_0}{\partial \tau} \right) \cos \tau d \tau .\]

        \item Use the above result to discuss the steady state and limit cycle when $f = (y^4 - 1) y'$.

      \end{enumerate}

    \end{problem}

    \begin{solution}
      \begin{enumerate}
        \item
          First, we will express $y$ using multiple time scales as
          \[ y = y(t_1, t_2) \]
          with $t_1 = t$ and $t_2 = \varepsilon t$. Taking derivatives we get
          \[ y' = \frac{\partial y}{\partial t_1} + \varepsilon \frac{\partial y}{\partial t_2} \]
          and
          \[ y'' = \frac{\partial^2 y}{\partial t_1^2} + 2 \varepsilon \frac{\partial^2 y}{ \partial
          t_1 \partial t_2} + \varepsilon^2 \frac{\partial^2}{\partial t_2^2} .\]
          Next, we expand $y$ asymptotically as
          \[ y = y_0(t_1,t_2) +\varepsilon y_1(t_1,t_2) + \dots .\]

          Plugging the derivatives of our expanded solution into the differential equation, we get
          the $O(1)$ equation
          \[ \frac{\partial^2 y_0}{\partial t_1^2} + y_0 = 0 .\]
          This ODE has solutions of the form
          \[ y_0 = a(t_2) \sin t_1 + b(t_2) \cos t_1 .\]
          Now we define
          \[ A(t_2) = \sqrt{a(t_2)^2 + b(t_2)^2} \]
          and
          \[ \phi(t_2) = \tan^{-1} \left( \frac{a(t_2)}{b(t_2)} \right) .\]
          Then
          \[ a(t_2) = A(t_2) \sin( \phi(t_2)) ,\]
          \[ b(t_2) = A(t_2) \cos( \phi(t_2)) .\]
          So we can write $y_0$ as
          \begin{align*}
            y_0 &= A(t_2) \sin(\phi(t_2)) \sin t_1 + A(t_1) \cos(\phi(t_2)) \cos t_1 \\
            &= A(t_2) \cos( t_1 + \phi(t_2))
          \end{align*}

        \item
          We see that by expanding $f$ in terms of its Taylor series we get
          \begin{align*}
            f(y, y') &= f \left (y_0 + \varepsilon y_1 + \dots, y_0' + \varepsilon y_1' + \dots \right) \quad \text{where derivatives are with respect to $t$} \\
            &= f \left( y_0, \frac{\partial y_0}{\partial t} \right)+ O(\varepsilon) \\
            &= f \left( y_0, \frac{\partial y_0}{\partial t_1} + \varepsilon \frac{\partial y_0}{\partial t_2} \right) \\
            &= f \left( y_0, \frac{\partial y_0}{\partial t_1} \right) + O(\varepsilon)
          \end{align*}

          Plugging in the derivatives of our expansion of $y$ and the Taylor series of $f$ into our equation, we get the $O(\varepsilon)$ equation
          \[ \frac{\partial^2 y_1}{\partial t_1^2} + 2 \frac{\partial^2 y_0}{\partial t_1 \partial t_2} + f \left( y_0, y_0' \right) + y_1 = 0 \]

          We already found $y_0$, so from here, we can calculate
          \begin{align*}
            \frac{\partial^2 y_0}{\partial t_2 \partial t_2} &= \frac{\partial^2}{\partial t_1 \partial t_2} \big( A(t_2) \cos(t_1 + \phi(t_2) \big) \\
            &= \frac{\partial}{\partial t_1} \left( \frac{\partial A}{\partial t_2} \cos(t_1 + \phi(t_2)) - A \frac{\partial \phi}{\partial t_2}
            \sin(t_1 + \phi(t_2)) \right) \\
            &= -\frac{\partial A}{\partial t_2} \sin(t_1 + \phi(t_2)) - A \frac{\partial \phi}{\partial t_2}
            \cos(t_1 + \phi(t_2)) \\
            &= -\frac{\partial A}{\partial t_2} \sin \tau - A \frac{\partial \phi}{\partial t_2}
            \cos \tau
          \end{align*}

          This simplifies our $O(\varepsilon)$ equation to
          \[ \frac{\partial^2 y_1}{\partial t_1^2} + y_1 = 2 \left( \frac{\partial A}{\partial t_2} \sin \tau + A \frac{\partial \phi}{\partial t_2}
          \cos \tau \right) - f \left( y_0, \frac{\partial y_0}{\partial t_1} \right) .\]

        \item
          The differential equation we have is on ODE in $t_1$. By avoiding secular terms we want to get a solution that is $2 \pi$-periodic in $t_1$.
          Assume $u$ is a solution to
          \[ \frac{\partial^2 u}{\partial t_1^2} + u = 0 .\]

          For convenience, we will write the operator $L = \frac{\partial^2}{\partial t_1^2} + 1$ and denote $(.,.)_{L^2(0,2\pi)}$ as $(.,)$. Then we have
          \begin{align*}
            (L y_1, u) &= (y_1, Lu) \quad \text{integrating by parts twice} \\
            &= (y_1, Au) \\
            &= (y_1, 0) \\
            &= 0
          \end{align*}
          So $y_1$ must be orthogonal in $L^2(0,2\pi)$ to all solutions of $Lu = 0$.

          We have that $\cos t_1$ and $\sin t_1$ form a basis for $\{ u | Lu=0\}$. We can add any phase to these basis functions, so we will use
          $\cos (t_1 + \phi(t_2))$ and $\sin (t_1 + \phi(t_2))$ as our basis. Using our expression for $L y_1$, we require that
          \[ 0 = \bigintss_{0}^{2 \pi} \left[ 2 \left( \frac{\partial A}{\partial t_2} \sin (t_1 + \phi(t_2)) + A \frac{\partial \phi}{\partial t_2}
          \cos( t_1 + \phi(t_2)) \right) - f \left( y_0, \frac{\partial y_0}{\partial t_1} \right) \right] \sin( t_1 + \phi(t_2)) dt_1 \]
          and
          \[ 0 = \bigintss_{0}^{2 \pi} \left[ 2 \left( \frac{\partial A}{\partial t_2} \sin (t_1 + \phi(t_2)) + A \frac{\partial \phi}{\partial t_2}
          \cos( t_1 + \phi(t_2)) \right) - f \left( y_0, \frac{\partial y_0}{\partial t_1} \right) \right] \cos( t_1 + \phi(t_2)) dt_1 \]
          Using the fact that $\int_{0}^{2 \pi} \cos^2 x dx = \int_{0}^{2 \pi} \sin^2 x dx = \pi$ and $\int_{0}^{2 \pi} \sin x \cos x dx = 0$, these
          conditions simplify to
          \[ \frac{\partial A}{\partial t_2} = \frac{1}{2 \pi} \int_{0}^{2 \pi} f \left( y_0, \frac{\partial y_0}{ \partial t_1} \right) \sin (t_1 +
          \phi(t_2)) dt_1 \]
          and
          \[ A \frac{\partial \phi}{\partial t_2} = \frac{1}{2 \pi} \int_{0}^{2 \pi} f \left( y_0, \frac{\partial y_0}{ \partial t_1} \right) \cos (t_1 +
          \phi(t_2)) dt_1 \]


        \item
      \end{enumerate}

    \end{solution}

  \item Holmes 3.35 \begin{problem}
      Consider the problem
      \[ \frac{d}{dt} \left( D(\varepsilon t) \frac{dy}{dt} \right) + y = 0 \quad \text{for } 0<t ,\]
      where $y(0) = \alpha$ and $y'(0)=\beta$. The coefficient $D(\tau)$ is a
      smooth positive function with $D' > 0$. Find a first term approximation
      of the solution valid for large $t$.
    \end{problem}

    \begin{solution}

      We assume that our solution can be written as $y = y(t_1, t_2)$ where $t_1 = f(t,\varepsilon)$ and $t_2 = \varepsilon t$ and $f$ satisfies:
      \begin{enumerate}
        \item $f$ is nonnegative and increases with $t$,
        \item $\varepsilon t << f$ as $\varepsilon \to 0$ ($t_1$ is the fast time scale and $t_2$ is the slow time scale), and
        \item $f$ is smooth
      \end{enumerate}

      By the chain rule, we then see
      \[ \frac{d y}{d t} = f_t \frac{\partial y}{\partial t_1} + \varepsilon \frac{\partial y}{\partial t_2} \]
      and
      \[ \frac{d^2 y}{dt^2} = f_t^2 \frac{\partial^2 y}{\partial t_1^2} + f_{tt} \frac{\partial y}{\partial t_1} + 2 \varepsilon f_t
      \frac{\partial^2 y}{\partial t_1 \partial t_2} + \varepsilon^2 \frac{\partial^2 y}{\partial t_2^2} .\]

      Plugging these derivatives into our equation, we get
      \[ D(\varepsilon t) \left( f_t^2 \frac{\partial^2 y}{\partial t_1^2} + f_{tt} \frac{\partial y}{\partial t_1} + 2 \varepsilon f_t
      \frac{\partial^2 y}{\partial t_1 \partial t_2} + \varepsilon^2 \frac{\partial^2 y}{\partial t_2^2} \right) + \varepsilon D'(\varepsilon t)
      \left( f_t \frac{\partial y}{\partial t_1} + \varepsilon \frac{\partial y}{\partial t_2} \right) + y = 0 .\]
      From this equation, we see that our rapid oscillations come from balancing the $ D(\varepsilon t) f_t^2 \frac{\partial^2 y}{\partial t_1^2}$ and
      $y$ terms. So we will let
      \[ f_t = \frac{1}{\sqrt{D(\varepsilon t)}} ,\]
      giving
      \[ f = \int_{0}^{t} \frac{1}{\sqrt{D(\varepsilon \tau)}} d \tau ,\]
      \[ f_{tt} = -\frac{\varepsilon}{2} (D(\varepsilon t))^{-3/2} D'(\varepsilon t) .\]
      Now we expand our solution asymptotically as
      \[ y(t_1,t_2) = y_0(t_1, t_2) + \varepsilon y_1(t_1,t_2) + \dots \]

      Plugging this expansion into our differential equation, we get the $O(1)$ equation
      \[ \frac{\partial^2 y_0}{\partial t_1^2} + y_0 = 0 \]
      with boundary conditions $y_0(0) = \alpha$ and $\frac{1}{\sqrt{D(0)}} \frac{\partial y_0}{\partial t_1} (0) = \beta$. (We are abusing notation
      slightly here; we use $y_0(0)$ to mean $y_0(0,0)$.)

      This has a solution
      \[ y_0 = A_0(t_2) \sin t_1 + B_0(t_2) \cos t_1 \]
      with $A_0(0) = \beta \sqrt{D(0)}$ and $B_0(0) = \alpha$.

      Moving on to the $O(\varepsilon)$ equation, we get
      \begin{align*}
        \frac{\partial^2 y_1}{\partial t_1^2} + y_1 &= \frac{1}{2} (D(\varepsilon t))^{-3/2} \frac{\partial y_0}{\partial t_1} -
        \frac{2}{\sqrt{D(\varepsilon t)}} \frac{\partial^2 y_0}{\partial t_1 \partial t_2} + D'(\varepsilon t) \frac{1}{\sqrt{D(\varepsilon t)}}
        \frac{\partial y_0}{\partial t_1} \\
        &= \left( \frac{1}{2} D'(\varepsilon t) (D(\varepsilon t))^{-3/2} + D'(\varepsilon t) \frac{1}{\sqrt{D(\varepsilon t)}} \right) (A_0(t_2) \cos t_1 - B_0(t_2)
        \sin t_1 ) \\
        &\quad - \frac{2}{\sqrt{D(\varepsilon t)}} \left( \frac{\partial A_0}{\partial t_2} \cos t_1 - \frac{\partial B_0}{\partial t_2} \sin t_1
        \right) \\
        &= \left( \frac{A_0}{2} D' D^{-3/2} + \frac{A_0 D'}{\sqrt{D}} - \frac{2}{\sqrt{D}} \frac{\partial A_0}{\partial t_2} \right) \cos t_1
        - \left( \frac{B_0}{2} D' D^{-3/2} + \frac{B_0 D'}{\sqrt{D}} - \frac{\partial B_0}{\partial t_2} \right) \sin t_1
      \end{align*}

      In order to prevent secular terms, we want
      \[ \frac{A_0}{2} D' D^{-3/2} + \frac{A_0 D'}{\sqrt{D}} - \frac{2}{\sqrt{D}} \frac{\partial A_0}{\partial t_2} = \frac{B_0}{2} D' D^{-3/2} +
      \frac{B_0 D'}{\sqrt{D}} - \frac{2}{\sqrt{D}} \frac{\partial B_0}{\partial t_2} = 0 .\]

      This first equation is
      \[ A_0' - \left( \frac{1}{4} D' D^{-1} + D' \right) A_0 = 0 .\]

      We introduce an integrating factor and find that
      \[ \left( A_0 \exp{ \left(-\frac{1}{4} \ln D - D \right) } \right) ' = 0 \]

      Simplifying this, we get
      \[ A_0 = C_1 D^{1/4} e^D . \]
      Similarly, we get
      \[ B_0 = C_2 D^{1/4} e^D .\]
      where $C_1$ and $C_2$ are constants. Using the initial conditions $A_0(0) = \beta \sqrt{D(0)}$ and $B_0(0) = \alpha$, we have
      \[ C_1 = \beta D(0)^{1/4} e^{-D(0)} \]
      and
      \[ C_2 = \alpha D(0)^{-1/4} e^{D(0)} .\]

      Therefore, our solution first term approximation is
      \[ y_0 = C_1 D(t_2)^{1/4}e^{D(t_2)} \sin t_1 + C_2 D(t_2)^{1/4} e^{D(t_2)} \cos t_1 .\]

    \end{solution}

  \item Holmes 4.19 \begin{problem}
      This problem concerns the Schr\"{o}dinger equation in 4.67 and the potential shown in Fig.
      4.8. Because $V(x)$ acts like a barrier, this example is commonly used to illustrate the
      phenomenon of tunneling.
      \begin{enumerate}
        \item Using 4.48 and 4.57, find a first-term approximation of the solution of 4.67 in the
          case of the barrier potential shown in fig. 4.8. Assume $E$ is given and satisfies
          $0<E<V_M$, where $V_M = \max_{-\infty < x < \infty} V(x)$ and $V(-\infty) = V(\infty) = 0$.

        \item The time-dependent Schr\"{o}dinger equation has the form
          \[ -\varepsilon^2 \partial_x^2 \Psi + V \Psi = i \partial_t \Psi .\]
          In regard to the barrier potential, suppose a wave approaches from the left. Because of
          reflection from the barrier, the solution should then consist of left and right travelling
          waves in the region $x<a$. However, part of the incident wave will be transmitted through
          the barrier and result in a right-running wave for $x>b$ (this is the phenomenon of
          tunneling). Assuming $\Psi = \exp(-iEt) \psi(x)$, use the results from part (a) to find
          first-term approximations of the waves in these two regions.
      \end{enumerate}
    \end{problem}

    \begin{solution}
      \begin{enumerate}
        \item
          This is a straightforward application of the WKB approximation. We define
          \[ q(x) = V(x) - E \]
          and see that $q(a) = q(b) = 0$ are the turning points with $q'(a) > 0$ and $q'(b) < 0$.

          We set
          \[ \theta_1(x) = \int_{x}^{a} \sqrt{|q(s)|} ds ,\]
          \[ \theta_2(x) = \int_{x}^{b} \sqrt{|q(s)|} ds ,\]
          \[ \kappa_1(x) = \int_{a}^{x} \sqrt{|q(s)|} ds ,\]
          and
          \[ \kappa_2(x) = \int_{b}^{x} \sqrt{|q(s)|} ds .\]

          By equation 4.48 in Holmes we get that
          \[ \psi \sim \begin{cases}
              \frac{1}{|q(x)|^{1/4}} \left[ 2 a_R \cos \left( \frac{1}{\varepsilon} \theta_1(x) - \frac{\pi}{4} \right) + b_R \cos \left(
              \frac{1}{\varepsilon} \theta_1(x) + \frac{\pi}{4} \right) \right] &\text{if } x < a, \\
              \frac{1}{q(x)^{1/4}} \left( a_R e^{-\kappa_1(x)/\varepsilon} + b_R e^{\kappa_1(x)/\varepsilon} \right) &\text{if } a < x < b
          \end{cases} \]

          Similarly, we can use 4.57 to get
          \[ \psi \sim \begin{cases}
              \frac{1}{q(x)^{1/4}} \left( A_L e^{\theta_2(x)/\varepsilon} + B_L e^{-\theta_2(x)/\varepsilon} \right) &\text{if } a < x < b \\
              \frac{1}{|q(x)|^{1/4}} \left[ 2 B_L \cos \left( \frac{1}{\varepsilon} \kappa_2(x) - \frac{\pi}{4} \right) + A_L \cos \left(
              \frac{1}{\varepsilon} \kappa_2(x) + \frac{\pi}{4} \right) \right] &\text{if } b < x
          \end{cases} \]

          For the region with $a < x < b$, we need the two approximations to be equal. That is
          \[ \frac{1}{q(x)^{1/4}} \left( a_R e^{-\kappa_1(x)/\varepsilon} + b_R e^{\kappa_1(x)/\varepsilon} \right) =
          \frac{1}{q(x)^{1/4}} \left( A_L e^{\theta_2(x)/\varepsilon} + B_L e^{-\theta_2(x)/\varepsilon} \right) \]
          for all $a < x < b.$ By plugging in $x=a$ and $x=b$, we get the pair of equations
          \begin{align*}
            a_R + b_R = A_L e^{\int_{a}^{b} \sqrt{|q(s)|} ds / \varepsilon} + B_L e^{-\int_{a}^{b} \sqrt{|q(s)|} ds / \varepsilon} \\
            a_R e^{-\int_{a}^{b} \sqrt{|q(s)|} ds / \varepsilon} + b_R e^{\int_{a}^{b} \sqrt{|q(s)|} ds / \varepsilon} = A_L + B_L
          \end{align*}

          Solving these equations, we get the relations
          \begin{align*}
            a_R &= A_L e^{\int_{a}^{b} \sqrt{|q(s)|} ds / \varepsilon} \\
            b_R &= B_L e^{-\int_{a}^{b} \sqrt{|q(s)|} ds / \varepsilon}
          \end{align*}

          Let
          \[ C = \int_{a}^{b} \sqrt{|q(s)|} ds / \varepsilon .\]
          Then we have
          \[ \psi \sim \begin{cases}
              \frac{1}{|q(x)|^{1/4}} \left[ 2 a_R \cos \left( \frac{1}{\varepsilon} \theta_1(x) - \frac{\pi}{4} \right) + b_R \cos \left(
              \frac{1}{\varepsilon} \theta_1(x) + \frac{\pi}{4} \right) \right] &\text{if } x < a \\
              \frac{1}{q(x)^{1/4}} \left( a_R e^{-\kappa_1(x)/\varepsilon} + b_R e^{\kappa_1(x)/\varepsilon} \right) &\text{if } a < x < b \\
              \frac{1}{|q(x)|^{1/4}} \left[ 2 e^C b_R \cos \left( \frac{1}{\varepsilon} \kappa_2(x) - \frac{\pi}{4} \right) + e^{-C} a_R \cos \left(
              \frac{1}{\varepsilon} \kappa_2(x) + \frac{\pi}{4} \right) \right] &\text{if } b < x
          \end{cases} \]

        \item
      \end{enumerate}

    \end{solution}

  \item Holmes 4.35 \begin{problem}
      The equation for a slender membrane with small damping is
      \[ \varepsilon^2 u_{xx} + u_{yy} = \mu^2(x) u_{tt} + \varepsilon \alpha(x) u_t
        \quad \text{ for }
        \begin{cases}
          0<x<\infty \\
          0<y<1,
      \end{cases} \]
      where $u(x,0,t) = u(x,1,t) = 0, u(0,y,t) = f(y) \cos(\omega t)$ and $u(x,y,t)$ is bounded as
      $x \to \infty$. Assume $\mu(x)$ and $\alpha(x)$ are positive and $\mu'<0$.

      \begin{enumerate}
        \item Use the WKB method to find an approximation of the long-time solution outside of the
          transition region. you only need to find the general solution for each mode.

        \item Assuming the turning point $x_t >0,$ find a first-term approximation in the transition
          layer and match it to the solutions from part (a). Assume it is a simple turning point.

      \end{enumerate}
    \end{problem}

    \begin{solution}

      \begin{enumerate}
          \item
            We begin by assuming our solution is of the form
            \[ u(x,y,t) = e^{i(\omega t - \theta(x)/\varepsilon)} (u_0(x,y) + \varepsilon u_1(x,y) + \dots ) .\]

            Now we must take the necessary derivatives:
            \[ u_x = e^{i(\omega t - \theta/\varepsilon)} \left( u_{0,x} + \varepsilon u_{1,x} + \dots \right) - \frac{i}{\varepsilon} \theta_x e^{i(\omega t - \theta/\varepsilon)} (u_0 + \varepsilon u_1 + \dots ) ,\]
            \begin{align*}
              u_{xx} &=  e^{i(\omega t - \theta/\varepsilon)} (u_{0,xx} + \varepsilon u_{1,xx} + \dots ) - \frac{2i}{\varepsilon} \theta_x e^{i(\omega t - \theta/\varepsilon)} (u_{0,x} + \varepsilon u_{1,x} + \dots ) \\
              &\quad - \frac{1}{\varepsilon^2} \theta_x^2 e^{i(\omega t - \theta/\varepsilon)} (u_0 + \varepsilon u_1 + \dots ) -\frac{i}{\varepsilon} \theta_{xx} e^{i(\omega t - \theta/\varepsilon)} (u_0 + \varepsilon u_1 + \dots) ,
            \end{align*}

            \[ u_y = e^{i(\omega t - \theta/\varepsilon)} (u_{0,y} + \varepsilon u_{1,y} + \dots ) ,\]

            \[ u_{yy} = e^{i(\omega t - \theta/\varepsilon)} (u_{0,yy} + \varepsilon u_{1,yy} + \dots ) ,\]

            \[ u_t = i\omega e^{i(\omega t - \theta/\varepsilon)} (u_0 + \varepsilon u_1 + \dots ) , \]
            and
            \[ u_{tt} = - \omega^2 e^{i(\omega t - \theta/\varepsilon)} (u_0 + \varepsilon u_1 + \dots ) .\]

            Next we need to plug these derivatives into our equation. We will skip writing this step explicitly and will instead write the necessary equations immediately. We begin with the $O(1)$ equation:
            \[ -\theta_x^2 u_0 + u_{0,yy} = -\mu^2 \omega^2 u_0 \]
            with the boundary conditions $u_0 = 0$ for $y = 0$ and $y = 1$.
            Rewriting this equation as
            \[ u_{0,yy} + (\mu^2 \omega^2 - \theta_x^2) u_0 = 0 ,\]
            we see it is an eigenvalue problem and is in fact the eikonal problem. As in the membrane problem from class, by setting $\lambda = \sqrt{\mu^2 \omega^2 - \theta_x^2}$, we get the solution
            \begin{equation} \label{eq:u_0}
               u_0(x,y) = A(x) \sin \left( \lambda y  \right) ,
            \end{equation}
            where we must have $\lambda = n \pi$ in order to satisfy the boundary conditions. Thus we have
            \[ \theta_x = \pm \sqrt{ \mu^2 \omega^2 - \lambda_n^2 } \text{ for } n = 1,2, \dots \]
            where $\lambda_n = n \pi$.

            Now we look at our $O(\varepsilon)$ equation:
            \[ -2i \theta_x u_{0,x} - \theta_x^2 u_1 - i \theta_{xx} u_0 + u_{1,yy} = -\mu^2 \omega^2 u_1 + i \omega \alpha u_0 ,\]
            which can be rewritten as
            \[ u_{1,yy} + \lambda_n^2 u_1 = i (\theta_{xx} u_0 + 2 \theta_x u_{0,x} + \omega \alpha u_0) \]
            with the boundary condition $u_1 = 0$ at $y = 0$ and $y = 1$.

            To get our solvability condition, we multiply by $u_0$ and integrate from 0 to 1 in the $y$ direction. This gives
            \[ \int_{0}^{1} u_0 (u_{1,yy} + \lambda_n^2 u_1) dy = i \int_{0}^{1} u_0 (\theta_{xx} u_0 + 2 \theta_x u_{0,x} + \omega \alpha u_0) dy \]
            For the left-hand side, integrating by parts twice leaves
            \begin{align*}
              \int_{0}^{1} u_0 (u_{1,yy} + \lambda_n^2 u_1) dy
              &= \int_{0}^{1} u_1 ( u_{0,yy} + \lambda_n^2 u_0) dy \\
              &= 0 \quad \parbox{4cm}{by the eikonal equation}
            \end{align*}

            So we have
            \begin{align*}
              0 &= \int_{0}^{1} u_0 (\theta_{xx} u_0 + 2 \theta_x u_{0,x} + \omega \alpha u_0) dy \\
              &= \int_{0}^{1} (\theta_x u_0^2)_x dy + \omega \alpha \int_{0}^{1} u_0^2 dy \quad \parbox{4cm}{($\omega, \alpha$ are constant in $y$)} \\
              &= \frac{d}{dx} \int_{0}^{1} \theta_x u_0^2 dy + \omega \alpha A^2 \int_{0}^{1} \sin^2 \left( n \pi y \right) dy \quad \parbox{4cm}{from equation \eqref{eq:u_0}} \\
              &= \frac{d}{dx} \theta_x \int_{0}^{1} u_0^2 dy + \frac{1}{2} \omega \alpha A^2 \\
              &= \frac{d}{dx} (\theta_x A^2) + \frac{1}{2} \omega \alpha A^2
            \end{align*}

            Therefore, we have
            \[ (A^2)' \theta_x + (\theta_{xx} + \frac{1}{2} \omega \alpha) A^2 = 0 \]
            We make the substitution $B = A^2$ to give the differential equation
            \[ B' \theta_x + (\theta_{xx} + \frac{1}{2} \omega \alpha) B = 0 .\]
            Rewriting, we have
            \[ B' + \frac{2 \theta_{xx} + \omega \alpha}{2 \theta_x} B = 0 .\]

            Using an integrating factor, we can see that
            \begin{align*}
              A &= \pm \sqrt{B} \\
              &= C e^{-\frac{1}{2} \int_{a}^{x} \frac{2 \theta_{xx} + \omega \alpha}{2 \theta_x} dx} \\
              &= \frac{C}{\sqrt{\theta_x}} e^{\frac{-1}{4} \int_{a}^{x} \frac{\omega \alpha}{\theta_x} dx}
            \end{align*}
            for a constant $C$. Our approximation is then given by
            \[ u_0(x,y) = A(x) \sin (\lambda_n y) \]
            with $A$ as above.

            For each mode, our approximation is then
            \[ u(x,y,t) \sim \frac{C}{\sqrt{\theta_x}} e^{-\frac{1}{4} \int_{a}^{x} \frac{\omega \alpha}{\theta_x} dx } e^{i (\omega t -
            \theta/\varepsilon)} \sin(\lambda n y) .\]

        \item
      \end{enumerate}

    \end{solution}
\end{enumerate}
\end{document}


