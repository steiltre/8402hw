%        File: hw4.tex
%     Created: Tue Apr 19 01:00 PM 2016 C
% Last Change: Tue Apr 19 01:00 PM 2016 C
%

\documentclass[a4paper]{article}

\title{Math 8402 Homework 4 }
\date{4/25/16}
\author{Trevor Steil}

\usepackage{amsmath}
\usepackage{amsthm}
\usepackage{amssymb}
\usepackage{esint}

\newtheorem{theorem}{Theorem}[section]
\newtheorem{corollary}{Corollary}[section]
\newtheorem{proposition}{Proposition}[section]
\newtheorem{lemma}{Lemma}[section]
\newtheorem*{claim}{Claim}
\newtheorem*{problem}{Problem}
%\newtheorem*{lemma}{Lemma}
\newtheorem{definition}{Definition}[section]

\newcommand{\R}{\mathbb{R}}
\newcommand{\N}{\mathbb{N}}
\newcommand{\C}{\mathbb{C}}
\newcommand{\Z}{\mathbb{Z}}
\newcommand{\supp}[1]{\mathop{\mathrm{supp}}\left(#1\right)}
\newcommand{\lip}[1]{\mathop{\mathrm{Lip}}\left(#1\right)}
\newcommand{\curl}{\mathrm{curl}}
\newcommand{\la}{\left \langle}
\newcommand{\ra}{\right \rangle}
\renewcommand{\vec}[1]{\mathbf{#1}}

\newenvironment{solution}{\emph{Solution.}}

\begin{document}
\maketitle
\begin{enumerate}
  \item Holmes 4.52
    \begin{problem}
      This problem examines what happens when $\mu$ is discontinuous across a smooth interface $Q$. The incident wave makes an angle $\varphi_i$ with
      $\mathbf{n}$, where $\mathbf{n}$ is the unit normal to $Q$ pointing to the incident side. It is assumed that $0 < \varphi_i <
      \frac{\pi}{2}$, and the resulting plane formed by $\mathbf{n}$ and the ray for $u_I$ determine what is called the plane of incidence. Also note
      that each wave has its own set of ray coordinates, which will be designated as $\mathbf{X}_I(s), \mathbf{X}_R(s),$ and $\mathbf{X}_T(s)$.
      Moreover, the tangent vectors $\partial_s \mathbf{X}_I(s_1), \partial_s \mathbf{X}_R(s_1)$, and $\partial_s \mathbf{X}_T(s_1)$, which are
      assumed to be nonzero, determine the angle of the respective wave with the normal to the surface. Finally, assume that $Q$ can be parametrized
      as $\mathbf{x} = \mathbf{r}(\alpha,\beta)$, where the tangent vectors $\mathbf{t}_\alpha = \partial_\alpha \mathbf{r}$ and $\mathbf{t}_\beta =
      \partial_\beta \mathbf{r}$ are orthonormal and $\mathbf{n} = \mathbf{t}_\alpha \times \mathbf{t}_\beta$.

      \begin{enumerate}
        \item Use the fact that $\theta_I(\mathbf{r}) = \theta_R(\mathbf{r})$ to show that $\nabla \theta_I - (\partial_n \theta_I) \mathbf{n} =
          \nabla \theta_R - (\partial_n \theta_R) \mathbf{n}$. With this and the eikonal equation, show that $\partial_n \theta_R = -\partial_n
          \theta_I$.

        \item The vector $\mathbf{n} \times \partial_s \mathbf{X}_I(s_1)$ is normal to the plane of incidence. Use this and the fact that
          $\theta_I(\mathbf{r}) = \theta_R(\mathbf{r})$ to show that $\partial_s \mathbf{X}_R(s_1)$ is in the plane of incidence. Moreover, show that
          $\cos \varphi_i = \cos \varphi_r$, and therefore $\varphi_i = \varphi_r$.

        \item Using an argument similar to that used in part (b), show that the ray for $u_T$ is in the plane of incidence and $\mu_+ \sin \varphi_i =
          \mu_- \sin \varphi_t$. This is known as Snell's law of refraction. Also, $\mu_+$ is the limiting value of $\mu$ when approaching $S$ from
          the incident side and $\mu_0$ the limiting value coming from the transmitted side.

        \item As shown in part (c), the transmitted angle is determined from the equation $\sin \varphi_t = (\mu_+/\mu_-) \sin \varphi_i$. If $\mu_+ >
          \mu_-$, then there are incident angles such that $(\mu_+/\mu_-) \sin \varphi_i > 1$, and this means there is no real-valued solution for the
          transmitted angle. In this case, show that $R = e^{-i\delta}$, where $0<\delta<\pi$. This produces a phase shift in the reflected wave that
          is associated with what is known as the Goos-H\"{a}nchen effect. Also show that the transmitted wave decays exponentially with distance from
          the interface; for this reason it is called an evanescent wave.

        \item If $\mu$ is constant, then the WKB approximation for the incident wave is given in (4.138). It is possible to write the reflected and
          transmitted waves in a similar form. Determine the reflected and transmitted wave vectors $k_R$ and $k_T$ in terms of the incident wave.
      \end{enumerate}
    \end{problem}

    \begin{solution}
      \begin{enumerate}
        \item

        \item

        \item

        \item

        \item

      \end{enumerate}

    \end{solution}

  \item Holmes 5.10
    \begin{problem}
      In the study of the diffusion of a disease through a population one finds the following problem:
      \begin{align*}
        \nabla \cdot (D \nabla S) - \beta SI &= 0, \\
        \nabla \cdot (D \nabla I) + \beta SI - \lambda I &= 0,
      \end{align*}
      where $\beta$ and $\lambda$ are positive constants. The variables $S$ and $I$ are the densities of susceptible and infected populations,
      respectively. Also, the function $D = D(x,x/\varepsilon)$ is assumed to be positive, smooth, and periodic in the fast coordinate. Find the
      homogenized version of this problem. Note that a somewhat different version of this problem is analyzed in Garlick et al. (2011).
    \end{problem}

    \begin{solution}
      We write our functions in the form $f = f(x,y)$. Then we have
      \begin{equation} \label{eqn:div_expansion}
        \nabla = \nabla_x + \frac{1}{\varepsilon} \nabla_y .
      \end{equation}

      We expand $I$ and $S$ as
      \begin{equation} \label{eqn:I_expansion}
        I = I_0(x,y) + \varepsilon I_1(x,y) + \dots
      \end{equation}
      and
      \begin{equation} \label{eqn:S_expansion}
        S = S_0(x,y) + \varepsilon S_1(x,y) + \dots
      \end{equation}

      Because we assumed $D$ is periodic in y, we will assume all terms in equations \eqref{eqn:I_expansion} and \eqref{eqn:S_expansion} are periodic
      in $y$ with the same period as $D$.

      Plugging \eqref{eqn:div_expansion}, \eqref{eqn:I_expansion}, and \eqref{eqn:S_expansion} into our first equation gives
      \begin{equation} \label{eqn:first_expansion}
        (\nabla_x + \frac{1}{\varepsilon} \nabla_y) \cdot (D (\nabla_x + \frac{1}{\varepsilon} \nabla_y) (S_0 + \varepsilon S_1 + \dots) - \beta (S_0
        + \varepsilon S_1 + \dots) (I_0 + \varepsilon I_1 + \dots) = 0.
      \end{equation}

      From here, we can find our $O(\frac{1}{\varepsilon^2})$ equation to be
      \begin{equation} \label{eqn:O(1/eps^2)}
        \nabla_y \cdot (D \nabla_y S_0) = 0.
      \end{equation}

      Multiplying by $S_0$ and integrating over $\Psi$, the periodic box determined by the period of $D$, we get
      \begin{align*}
        0 &= \int_{\Psi}^{} S_0 \nabla_y \cdot (D \nabla_y S_0) dy \\
        &= \int_{\Psi}^{} D |\nabla_y S_0|^2 dy \quad \parbox{5cm}{by integrating by parts and using periodicity}
      \end{align*}

      Therefore, we know $|\nabla_y S_0|=0$ because $D>0$. So we know $S_0 = S_0(x)$ is independent of $y$.

      Now from equation \eqref{eqn:first_expansion}, we get the $O(\frac{1}{\varepsilon})$ equation
      \begin{equation}
        \nabla_y \cdot (D \nabla_x S_0) + \nabla_x \cdot (D \nabla_y S_0) + \nabla_y \cdot (D \nabla_y S_1) = 0
        \label{eqn:O(1/eps)}
      \end{equation}

      Because $S_0$ is independent of $y$, this simplifies to
      \begin{equation}
        \nabla_y \cdot( D \nabla_y S_1) = - \nabla_y D \cdot \nabla_x S_0
        \label{eqn:simp_O(1/eps)}
      \end{equation}

      Let $\{ e_i \}$ denote the standard basis on $\R^2$. For $k = 1,2$, let $a_k$ satisfy
      \begin{equation}
        \begin{cases}
          \nabla_y \cdot (D \nabla_y a_k) = - e_k \cdot \nabla_y D = - \frac{\partial D}{\partial y_k} \\
          \int_{\Psi}^{} a_k dx = 0
        \end{cases}
        \label{eqn:a_k}
      \end{equation}

      Then we have
      \begin{equation}
        S_1 = (\nabla_x S_0) \cdot a + c_1(x)
        \label{eqn:S_1}
      \end{equation}
      where
      \begin{equation} \label{eqn:a}
          a = \begin{pmatrix}
            a_1 \\
            a_2
        \end{pmatrix}
      \end{equation}
      and $c_1$ is some function of $x$.

      From equation \eqref{eqn:first_expansion}, we get the $O(1)$ equation
      \begin{equation*}
        \nabla_x \cdot (D \nabla_x S_0) + \nabla_y \cdot (D \nabla_x S_1) + \nabla_x \cdot (D \nabla_y S_1) + \nabla_y \cdot (D \nabla_y S_2) - \beta S_0 I_0 = 0
      \end{equation*}

      We can rewrite this as
      \begin{equation}
        \nabla_x \cdot \left[ D ( \nabla_x S_0 + \nabla_y S_1) \right] + \nabla_y \cdot \left[ D ( \nabla_x S_1 + \nabla_y S_2) \right] - \beta S_0
        I_0 = 0
        \label{eqn:O(1)}
      \end{equation}

      Then by periodicity and the divergence theorem
      \begin{align*}
        \int_{\Psi}^{} \nabla_y \cdot \left[ D( \nabla_x S_1 + \nabla_y S_2) \right] dy = 0
      \end{align*}

      Using Einstein summation notation, we calculate
      \begin{align*}
        \nabla_x \cdot \left[ D (\nabla_x S_0 + \nabla_y S_1) \right] &= \nabla_x \cdot \left[ D ( \nabla_x S_0 + \nabla_y ( (  \nabla_x S_0 ) \cdot
        a) ) \right] \\
        &= \frac{\partial}{\partial {x_i}} \left( D \left( \frac{\partial}{\partial {x_i}} S_0 + \frac{\partial}{\partial {y_i}} \left( \frac{\partial
        S_0}{\partial x_j} a_j \right) \right) \right) \\
        &= \frac{\partial}{\partial x_i} \left( D \left( \frac{\partial}{\partial x_i} S_0 + \frac{\partial S_0}{\partial x_j} \frac{\partial
        a_j}{\partial y_i} \right) \right) \\
        &= \frac{\partial}{\partial x_i} \left( D \left( \delta_{ij} + \frac{\partial a_j}{\partial y_i} \right) \frac{\partial S_0}{\partial x_j}
        \right)
      \end{align*}

      We define $\hat{D}$ by
      \begin{equation}
        \hat{D}_{ij} = \frac{1}{|\Psi|} \int_{\Psi}^{} D \left( \delta_ij + \frac{\partial a_j}{\partial y_i} \right) dy
        \label{eqn:D_ij}
      \end{equation}

      Averaging equation \eqref{eqn:O(1)} over $\Psi$ we get our first homogenized equation
      \begin{equation}
        \la \beta S_0 I_0 \ra = \frac{\partial}{\partial x_i} \hat{D}_{ij} \frac{\partial S_0}{\partial x_j} = \nabla \cdot (\hat{D} \nabla S_0 )
        \label{eqn:first_hom}
      \end{equation}
      where
      \[ \la f \ra = \frac{1}{|\Psi|} \int_{\Psi}^{} f dy .\]

      We perform similar steps for the second equation. By plugging \eqref{eqn:div_expansion}, \eqref{eqn:I_expansion}, and \eqref{eqn:S_expansion}
      into our second equation, we get
      \begin{equation}
        (\nabla_x + \frac{1}{\varepsilon} \nabla_y) \cdot ( D (\nabla_x + \frac{1}{\varepsilon} \nabla_y) (I_0 + \varepsilon I_1 + \dots) + \beta (S_0
        + \varepsilon S_1 + \dots) (I_0 + \varepsilon I_1 + \dots) - \lambda (I_0 + \varepsilon I_1 + \dots) = 0
        \label{eqn:second_expansion}
      \end{equation}

      From here we can find our $O(\frac{1}{\varepsilon^2})$ equation to be
      \begin{equation}
        \nabla_y \cdot(D \nabla_y I_0) = 0
        \label{eqn:second_O(1/eps^2)}
      \end{equation}

      Just as before, this shows us that $I_0 = I_0(x)$ is independent of $y$.

      From equation \eqref{eqn:second_expansion}, we get the $O(\frac{1}{\varepsilon})$ equation, we get
      \begin{equation}
        \nabla_x \cdot ( D \nabla_y I_0) + \nabla_y \cdot ( D \nabla_x I_0 ) + \nabla_y \cdot ( D \nabla_y I_1) = 0
        \label{eqn:second_O(1/eps)}
      \end{equation}

      As before, we can use this to write
      \begin{equation}
        I_1 = (\nabla_x I_0) \cdot a + c_2(x)
        \label{eqn:I_1}
      \end{equation}
      with $a$ defined by equations \eqref{eqn:a_k} and \eqref{eqn:a}.

      From equation \eqref{eqn:second_expansion}, we get the $O(1)$ equation
      \begin{equation}
        \nabla_x \cdot \left[ D( \nabla_x I_0 + \nabla_y I_1 ) \right] + \nabla_y \cdot \left[ D ( \nabla_x I_1 + \nabla_y I_2 ) \right] + \beta S_0
        I_0 - \lambda I_0 = 0
        \label{eqn:second_O(1)}
      \end{equation}

      Then using periodicity and the divergence theorem again, we get
      \[ \int_{\Psi}^{} \nabla_y \cdot \left[ D ( \nabla_x I_1 + \nabla_y I_2 ) \right] dy = 0 .\]
      By averaging equation \eqref{eqn:second_O(1)} over $\Psi$ in the same way as for our first equation, we get the second homogenized equation
      \begin{equation}
        \la \beta S_0 I_0 - \lambda I_0 \ra = \nabla \cdot ( \hat{D} \nabla I_0 )
        \label{eqn:second_hom}
      \end{equation}
    \end{solution}

\end{enumerate}
\end{document}


