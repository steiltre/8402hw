%        File: hw4.tex
%     Created: Tue Apr 19 01:00 PM 2016 C
% Last Change: Tue Apr 19 01:00 PM 2016 C
%

\documentclass[a4paper]{article}

\title{Math 8402 Homework 4 }
\date{4/25/16}
\author{Trevor Steil}

\usepackage{amsmath}
\usepackage{amsthm}
\usepackage{amssymb}
\usepackage{esint}

\newtheorem{theorem}{Theorem}[section]
\newtheorem{corollary}{Corollary}[section]
\newtheorem{proposition}{Proposition}[section]
\newtheorem{lemma}{Lemma}[section]
\newtheorem*{claim}{Claim}
\newtheorem*{problem}{Problem}
%\newtheorem*{lemma}{Lemma}
\newtheorem{definition}{Definition}[section]

\newcommand{\R}{\mathbb{R}}
\newcommand{\N}{\mathbb{N}}
\newcommand{\C}{\mathbb{C}}
\newcommand{\Z}{\mathbb{Z}}
\newcommand{\supp}[1]{\mathop{\mathrm{supp}}\left(#1\right)}
\newcommand{\lip}[1]{\mathop{\mathrm{Lip}}\left(#1\right)}
\newcommand{\curl}{\mathrm{curl}}
\newcommand{\la}{\left \langle}
\newcommand{\ra}{\right \rangle}
\renewcommand{\vec}[1]{\mathbf{#1}}

\newenvironment{solution}{\emph{Solution.}}

\begin{document}
\maketitle
\begin{enumerate}
  \item Holmes 4.52
    \begin{problem}
      This problem examines what happens when $\mu$ is discontinuous across a smooth interface $Q$. The incident wave makes an angle $\varphi_i$ with
      $\mathbf{n}$, where $\mathbf{n}$ is the unit normal to $Q$ pointing to the incident side. It is assumed that $0 < \varphi_i <
      \frac{\pi}{2}$, and the resulting plane formed by $\mathbf{n}$ and the ray for $u_I$ determine what is called the plane of incidence. Also note
      that each wave has its own set of ray coordinates, which will be designated as $\mathbf{X}_I(s), \mathbf{X}_R(s),$ and $\mathbf{X}_T(s)$.
      Moreover, the tangent vectors $\partial_s \mathbf{X}_I(s_1), \partial_s \mathbf{X}_R(s_1)$, and $\partial_s \mathbf{X}_T(s_1)$, which are
      assumed to be nonzero, determine the angle of the respective wave with the normal to the surface. Finally, assume that $Q$ can be parametrized
      as $\mathbf{x} = \mathbf{r}(\alpha,\beta)$, where the tangent vectors $\mathbf{t}_\alpha = \partial_\alpha \mathbf{r}$ and $\mathbf{t}_\beta =
      \partial_\beta \mathbf{r}$ are orthonormal and $\mathbf{n} = \mathbf{t}_\alpha \times \mathbf{t}_\beta$.

      \begin{enumerate}
        \item Use the fact that $\theta_I(\mathbf{r}) = \theta_R(\mathbf{r})$ to show that $\nabla \theta_I - (\partial_n \theta_I) \mathbf{n} =
          \nabla \theta_R - (\partial_n \theta_R) \mathbf{n}$. With this and the eikonal equation, show that $\partial_n \theta_R = -\partial_n
          \theta_I$.

        \item The vector $\mathbf{n} \times \partial_s \mathbf{X}_I(s_1)$ is normal to the plane of incidence. Use this and the fact that
          $\theta_I(\mathbf{r}) = \theta_R(\mathbf{r})$ to show that $\partial_s \mathbf{X}_R(s_1)$ is in the plane of incidence. Moreover, show that
          $\cos \varphi_i = \cos \varphi_r$, and therefore $\varphi_i = \varphi_r$.

        \item Using an argument similar to that used in part (b), show that the ray for $u_T$ is in the plane of incidence and $\mu_+ \sin \varphi_i =
          \mu_- \sin \varphi_t$. This is known as Snell's law of refraction. Also, $\mu_+$ is the limiting value of $\mu$ when approaching $S$ from
          the incident side and $\mu_0$ the limiting value coming from the transmitted side.

        \item As shown in part (c), the transmitted angle is determined from the equation $\sin \varphi_t = (\mu_+/\mu_-) \sin \varphi_i$. If $\mu_+ >
          \mu_-$, then there are incident angles such that $(\mu_+/\mu_-) \sin \varphi_i > 1$, and this means there is no real-valued solution for the
          transmitted angle. In this case, show that $R = e^{-i\delta}$, where $0<\delta<\pi$. This produces a phase shift in the reflected wave that
          is associated with what is known as the Goos-H\"{a}nchen effect. Also show that the transmitted wave decays exponentially with distance from
          the interface; for this reason it is called an evanescent wave.

        \item If $\mu$ is constant, then the WKB approximation for the incident wave is given in (4.138). It is possible to write the reflected and
          transmitted waves in a similar form. Determine the reflected and transmitted wave vectors $k_R$ and $k_T$ in terms of the incident wave.
      \end{enumerate}
    \end{problem}

    \begin{solution}
      \begin{enumerate}
        \item

        \item

        \item

        \item

        \item

      \end{enumerate}

    \end{solution}

  \item Holmes 5.10
    \begin{problem}
      In the study of the diffusion of a disease through a population one finds the following problem:
      \begin{align*}
        \nabla \cdot (D \nabla S) - \beta SI &= 0, \\
        \nabla \cdot (D \nabla I) + \beta SI - \lambda I &= 0,
      \end{align*}
      where $\beta$ and $\lambda$ are positive constants. The variables $S$ and $I$ are the densities of susceptible and infected populations,
      respectively. Also, the function $D = D(x,x/\varepsilon)$ is assumed to be positive, smooth, and periodic in the fast coordinate. Find the
      homogenized version of this problem. Note that a somewhat different version of this problem is analyzed in Garlick et al. (2011).
    \end{problem}

    \begin{solution}
    \end{solution}

\end{enumerate}
\end{document}


